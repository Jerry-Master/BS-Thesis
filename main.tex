\documentclass[12pt]{report}
\usepackage[utf8]{inputenc}
\usepackage[T1]{fontenc}
\usepackage[style=english]{csquotes}
\usepackage[english, spanish, catalan]{babel}
\usepackage{graphicx}
\usepackage{geometry}
\PassOptionsToPackage{hyphens}{url}\usepackage{hyperref}
\usepackage{titlesec}
\usepackage{lipsum}
\usepackage{fancyhdr}
\usepackage{amsmath}
\usepackage{amssymb}
\usepackage{tikz-cd}
\usepackage{subcaption}
\usepackage{stmaryrd}
\usepackage{url}
\usepackage{tikz}
\usetikzlibrary{shapes,
                tikzmark}
\tikzset{every tikzmarknode/.style={%
        draw=red, semithick, inner sep=2pt}
        }
\usepackage{forest}
\usetikzlibrary{fit,positioning}

\tikzset{
  red arrow/.style={
    midway,red,sloped,fill, minimum height=3cm, single arrow, single arrow head extend=.5cm, single arrow head indent=.25cm,xscale=0.3,yscale=0.15,
    allow upside down
  },
  black arrow/.style 2 args={-stealth, shorten >=#1, shorten <=#2},
  black arrow/.default={1mm}{1mm},
  tree box/.style={draw, rounded corners, inner sep=0.5em},
  node box/.style={white, draw=black, text=black, rectangle, rounded corners},
}
\input{definitions.tex}

%% Bibliography
\usepackage[backend=biber,style=numeric]{biblatex}
\DefineBibliographyStrings{english}{
  bibliography = {References},
}
\addbibresource{references.bib}

%% Cover, abstract and page style
%% Set up the geometry and hyperref packages
\geometry{
  a4paper,
  total={170mm,257mm},
  left=20mm,
  top=20mm,
}

\hypersetup{
    colorlinks=true,
    linkcolor=black,
    filecolor=magenta,
    urlcolor=cyan,
}

%% Define the cover page
\newcommand{\createcover}{
  \begin{titlepage}
    \centering
    \includegraphics[width=0.3\textwidth]{logos/logo_FIB.png}\hspace{0.1cm}
    \includegraphics[width=0.3\textwidth]{logos/logo_ETSETB.png}\hspace{0.1cm}
    \includegraphics[width=0.25\textwidth]{logos/logo_FME.png}\\[1cm]
    
    {\large\bfseries BACHELOR'S DEGREE IN MATHEMATICS \\[0.3cm]  BACHELOR'S DEGREE IN DATA SCIENCE AND ENGINEERING}\\[1cm]
    
    {\huge\bfseries Automated detection of tumoural cells with graph neural networks}\\[1cm]

    \includegraphics[width=0.7\textwidth]{imgs/graph_overlay.png}\\[2cm]
    
    {\large\itshape Author: Jose Pérez Cano}\\
    {\large\itshape Supervisor: Philippe Salembier Clairon}\\
    {\large\itshape Co-director: Ferran Marqués Acosta}\\[0.5cm]
    
    {\large June 2023}\\
    \vfill
    \includegraphics[width=0.3\textwidth]{logos/logo_CFIS.jpg}
  \end{titlepage}
}

%% Define the abstract page
\newcommand{\createabstract}[4]{
\selectlanguage{#1}
\begin{abstract}
    \noindent #2 % Abstract text.
    \noindent \\
    \noindent #3 % Keywords.
    \noindent \\
    \noindent #4 % AMS codes
\end{abstract}
\selectlanguage{english}
}

%% Set up the headers and footers
\pagestyle{fancy} % Line on top of sections
\fancyhf{}
\renewcommand{\chaptermark}[1]{\markboth{#1}{}} 
\renewcommand{\sectionmark}[1]{\markright{\thesection\ #1}}
\fancyhead[L]{Jose Pérez Cano}
\fancyhead[R]{\leftmark}
\fancyfoot[C]{\thepage}
\fancypagestyle{plain}{
  \fancyhead{}
  \renewcommand{\headrulewidth}{0pt}
}

%% Customize the chapter and section formatting
\titleformat{\chapter}[display]
  {\normalfont\huge\bfseries}{\chaptertitlename\ \thechapter}{20pt}{\Huge}
\titlespacing*{\chapter}{0pt}{-30pt}{40pt}

%% Assemble the document
\begin{document}

\setlength{\headheight}{14.49998pt}
\addtolength{\topmargin}{-2.49998pt}

\createcover
\pagenumbering{roman}
\createabstract{english}{The detection of tumoural cells from whole slide images is an essential task in medical diagnosis and research. In this thesis, we propose and analyse a novel approach that combines vision-based models with graph neural networks to improve the accuracy of automated tumoural cell detection. Our proposal leverages the inherent structure and relationships between cells in the tissue. Experimental results on our own curated dataset shows that several different metrics improve by up to $15\%$ compared to just using the vision approach. It has been proved to work with H\&E stained lung tissue and HER2 stained breast tissue. We believe that our proposed method has the potential to improve the accuracy of automated tumoural cell detection, which can lead to accelerated diagnosis and research in the field by reducing the worload of hystopathologists.}{\textbf{Keywords:} Histology, Lung, Breast, Graph Neural Networks, Convolutional Neural Networks, Calibration.}{\textbf{AMS Codes:} 92C50, 92C55, 92C37, 68T10, 68T45, 68R10}
\clearpage
\createabstract{spanish}{La detección de células tumorales en imágenes de portaobjeto completo juega un papel esencial en el diagnóstico médico y es un elemento fundamental de la investigación sobre el cáncer. En esta tesis proponemos y analizamos un enfoque novedoso que combina modelos de visión por ordenador con redes neuronales en grafos para mejorar la precisión de la detección automatizada de células tumorales. Nuestra propuesta aprovecha la estructura inherente y las relaciones entre las células del tejido. Los resultados experimentales obtenidos sobre nuestra propia base de datos muestran que varias métricas mejoran hasta en un 15\% en comparación con solo usar el enfoque de visión. Se ha demostrado que funciona con tejido pulmonar teñido con H\&E y tejido mamario teñido con HER2. Creemos que nuestro método tiene el potencial de mejorar la precisión de los métodos automáticos de detección de células tumorales, lo que puede llevar a acelerar los diagnósticos y la investigación en este ámbito al reducir la carga de trabajo de los histopatólogos.}{\textbf{Palabras clave:} Histología, Pulmón, Mama, Redes Neuronales en Grafos, Redes Neuronales Convolucionales, Calibración.}{\textbf{Códigos AMS:} 92C50, 92C55, 92C37, 68T10, 68T45, 68R10}
\clearpage
\createabstract{catalan}{La detecció de cèl·lules tumorals en imatges de seccions completes és una tasca essencial en el diagnòstic mèdic i la investigació. En aquesta tesi, proposem i analitzem un enfocament innovador que combina models basats en visió amb xarxes neuronals en grafs per millorar la precisió de la detecció automatitzada de cèl·lules tumorals. La nostra proposta aprofita l'estructura inherent i les relacions entre cèl·lules en el teixit. Els resultats experimentals en el nostre propi conjunt de dades curat mostrin que diversos indicadors milloren fins a un 15\% en comparació amb només usar l'enfocament de visió. S'ha demostrat que funciona amb teixit pulmonar tenyit amb H\&E i teixit mamari tenyit amb HER2. Creiem que el nostre mètode proposat té el potencial de millorar la precisió de la detecció automatitzada de cèl·lules tumorals, el que pot portar a uns diagnòstics més ràpids i una investigació accelerada en el camp degut a la reducció en la càrrega de treball dels histopatòlegs.}{\textbf{Paraulas clau:} Histologia, Pulmó, Mama, Xarxes Neuronals en Grafs, Xarxes Neuronals Convolucionals, Calibració.}{\textbf{Codis AMS:} 92C50, 92C55, 92C37, 68T10, 68T45, 68R10}
\tableofcontents
\clearpage
\pagenumbering{arabic}

\chapter{Introduction}
\section{When technology meets histology}

This project was born due to the necessity of alleviating physicians and researchers workload in the field of digital pathology. On a typical day, our experts have to go through a wide variety of tissue images in order to detect some anomaly or disease. Depending on the task at hand they sometimes need to estimate the proportion of tumoural cells with respect to healthy ones. How wonderful would it be if a machine could perform that task for them. That is the purpose of this thesis, to automatically estimate the percentage of tumoural cells with respect to non-tumoural ones and facilitate the physician's job.

Analysing human tissue is a challenging task. The first part of the process consists of extracting the tissue from the part of the body that is relevant to the patient's condition. If the patient is alive the extraction is typically done using a needle. Otherwise, if the organ is already removed then a cube of tissue can be sliced. Then, simply watching those slices through a microscope is not going to be enough. Cells are very small and its interior is difficult to observe even when looking through the lenses of a microscope. For that reason tissue is stained prior to observing it. There are different kind of staining, some of them highlight the cell nuclei, others the membrane and other stainings react with specific kind of cells. Another factor to take into account is the cost of the dye. Some of them make the task at hand easier but are too expensive to do for every patient. A trade-off is usually found where the more expensive one is used only when the cheaper is not enough for confidently diagnosing.

In the past, the surgically extracted slices were typically watched through the lenses of a microscope \cite{Chen2011}, what is termed as optical microscopy. With the development of new technologies, the field has become more and more digitalised \cite{Kumar2020}, now being called digital pathology. The resulting digital images that originate from watching through the microscope the tissue are called Whole Slide Images (WSI). Between when a biopsy procedure is made and when the specialist watches it, there is now a period of time required to digitalise it. In other words, at the morning one specialist carries out the removal of the tissue. Afterwards, technicians digitalise the image in the afternoon. It is in the next morning the physician watches the result. Taking advantage of that interval between when the image is digitised and when the doctor watches it, other computational methods can be applied prior to the experts receiving the images. This will enhance the physician user experience while working without the need for them to wait for the algorithm to give the result at real time because it is already precomputed. Having a preliminary diagnosis can help reduce the workload and make the histologist more productive.

\section{Why graphs?}

This whole thesis was initially thought to be about lung tissue. Starting from a WSI we are interested in detecting which cells are tumoural and which cells are healthy. The purpose of the application is to rapidly detect tissue slices that contain a high amount of tumoural DNA so that it can be later on processed and analysed. Finding such WSI requires the histologist to look at several of them and deciding which one to choose. So we want to provide a ranking of images from more likely of having a high percentage of tumour to less likely. This way, on average, the physician would require to look at less images per patient, making it possible to analyse more patients' WSI in the same time. So, where does graphs come into play?

In a previous thesis, the same problem was tackled using a computer vision-only approach \cite{upcommons353765}. Initially, I was asked to improve on that method. After several months working on the problem I began to notice the principal flaw (which is also the principal feature) of convolutional neural networks: an inductive bias towards locality \cite{DBLP:journals/corr/CohenS16a}. That means deep neural networks classify cells based on their immediate morphological properties. However, having met with pathologist Irene Sansano Valero, which is an expert histologist in the field of lung tumour, made it clear that cells were considered being tumoural or not depending on their surroundings. Visually identical cells may be classified differently if their neighbourhood is different. Here, I am always referring to lung tissue, we will later on analyse if this still holds true for other tissues. It was made clear then that a different approach was needed. Another kind of inductive bias was required, a relational inductive bias. One that classified cells based on their relationships with nearby cells and not only on their individual properties. This is the exact kind of inductive bias graph neural networks provide \cite{DBLP:journals/corr/abs-2104-13478}. 

This is how the idea came into existence. The exact details of how the graph is constructed and which networks are employed are later discussed. But the key idea is that if relations between cells are considered important when classifying them, using graphs in some way may lead to better results. We expect this hypothesis to be specially true for lung tissue and, in fact, we come across high evidence that it is so. The question then is, is it true for other tissues? The answer is it depends. Apart from the lung dataset we repeat all the experiments for other three datasets obtaining mixed results. Sometimes it is a good fit, sometimes it is not and sometimes it works better but not because it is a graph but because it is a stacking classifier. The exact meaning of these words and which experiments are made to prove it will be later discussed.
\chapter{Problem Formulation and State Of The Art}

\section{Definition}

In order to estimate the percentage of tumoural cells we decided to solve other problem. Instead of just predicting a number from a WSI, which would make the problem a logistic regression problem, we decided to segment and classify every cell to later count them and compute the percentage. There are two reasons why we chose to do so. The first one is because it makes models more interpretable, and the second one is because it makes the problem easier to solve.

Developing more on that first reason, we work in the medical field. Here, having a model that is statistically better than, say, a student is not enough to consider it really better than the student. The student can explain itself on why it made that diagnosis and so it provides more insight on how and why it makes mistakes. A seemingly black box model that simply outputs a number (the percentage of tumoural cells) gives no insight whatsoever of how and when you may expect it to fail. This is very dangerous in a medical setting. If we were at a factory classifying packages, we don't care so much about that. We can recover lost packages later on while still benefiting from the efficiency of using a automated process. However, we cannot recover dead patients. That makes interpretability a must. If our model predicts cells we can see which cells are causing more trouble. This helps know which data is needed to used for retraining the model and fixing those mistakes in the future. In the end, we want a model that can learn from its mistakes, looking at individual cell predictions makes it easier to recognise patterns in the mistakes it is doing and so it makes it easier to incrementally improve the model from its errors.

The second reason is that the regression problem is harder to solve. It may seem to be the other way around but it is not. Having worked with machine learning problems for years, my experience is that a regression problem where images are the input is almost never a good idea. Reframing the problem to solve something apparently harder in between has brought me better results in the past. Nonetheless, let's describe mathematically each problem in order to analyse each of them and come to an intuition of why it is a bad idea.

First, let's denote by $\mathcal{X}$ and $\mathcal{Y}$ the input and output space of the problem. We expect to find a function $f : \mathcal{X} \to \mathcal{Y}$ that effectively predicts the correct percentage given the WSI. Any WSI is no more than a collection of pixels, for that reason they can be viewed as very high dimensional vectors $\mathcal{X} = \mathbb{R}^N$, with $N > 10^{9}$ \cite{DICOM}. Similarly, the percentage is just a number between 0 and 1, so $\mathcal{Y} = [0,1]$.

Now, the logistic regression approach consists of finding such function from a family of parametric models $\mathcal{F}_\theta = \{ f_\theta | f_\theta : \mathcal{X} \to \mathcal{Y} \}$ while the segmentation approach tries to find $f$ differently. It first divides the WSI into patches. Then, each patch is processed independently to obtain pixel-wise predictions which are then used to compute the percentage. So, the family of parametric models now in consideration is $\mathcal{G}_\theta = \{ g_\theta | g_\theta : \mathcal{X}_s \to \mathcal{X}_s,\ \mathcal{X}_s \subset \mathcal{X} \}$ where $\mathcal{X}_s$ represents the space of images of 1024 by 1024 pixels, meaning $\text{dim}(\mathcal{X}_s) \approx 3.1 \cdot 10^6$. Given any segmentation model $g_\theta$ we construct its regression counterpart $f_\theta$ as described in the following commutative diagram. 

\[ \begin{tikzcd}
\mathcal{X} \arrow{r}{f_\theta} \arrow[swap]{d}{s} & \mathcal{Y} \\%
\mathcal{X}_s^P \arrow{r}{\tilde{g}_\theta}& \mathcal{X}_s^P \arrow{u}{c}
\end{tikzcd}
\]

Where we have extended $g_\theta$ to several patches by applying it independently to all of them.

\begin{align}
\begin{split}
\tilde{g}_\theta : \mathcal{X}_s^P& \to \mathcal{X}_s^P\\
(x_1, ..., x_P)& \mapsto (g_\theta(x_1), ..., g_\theta(x_P))
\end{split}
\end{align}

And the $s$ and $c$ functions refer to the split and count operations. Given a WSI it is split into several patches, for each patch, every cell is segmented and then all the tumoural and healthy cells are counted.

Without making any further assumptions about the families of functions we can derive some important insights about the advantages and disadvantages of each approach. First of all, it is clear that in the first approach the models have access to a wider context. Considering independent patches limits the ability to take into account global information. For instance, in lung tissue there is a structure called the cilium. It is a filamentous structure that appears near bronchioles. Its own existence is reason enough for considering the nearby cells as healthy. An example of such structure is depicted on \autoref{fig:lung-cilium}. The presence of cilium in the middle of a WSI cannot be taken into account when classifying the border of the tissue. But this disadvantage is not such a big deal. In a 1024 by 1024 image there is room enough for all the cells affected by the cilium, so limiting to such patches is enough. There is also another type of cell that interacts with their surrounding, the erythrocyte, also called red blood cell. All the cells glued to it are considered healthy. But that only applies to very close cells to it, so by using patches you will have no problem with this type of interactions. On the other hand, if we look at the dimensionality of input and output spaces we see a clear difference between the two approaches. For $\mathcal{F}_\theta$ the input space has a huge dimensionality while the output is uni-dimensional. In contrast, for $\mathcal{G}_\theta$ the input and output space both have the same dimensionality which is several orders of magnitude smaller than the dimensionality of $\mathcal{X}$. This makes the first approach quite prone to the curse of dimensionality \cite{AnalyticsVidhya}\footnote{Notice that this is not true for other regression problems, NeRFs \cite{mildenhall2020nerf} need to increase the dimensionality of the input to work properly.}. Even worse, we have just one number per each WSI. Even a simple regression problem needs more than 30 samples to have a reasonable amount of uncertainty. Having such amount of WSI is a very difficult task.

One may also think that the regression problem can be split into patches too. That is, solving the problem for individual patches and then averaging the percentages. This way the difference in dimensionality of the input and output spaces is lower. Moreover, the data scarcity problem is solved. But it is not so easy. In my experience, even with 224 by 224 images, regression is quite unfeasible if done in a naive way. On the other hand, for the individual percentages of each patch to be enough we will need to label all the patches of a WSI. Otherwise the sampling may not be uniform enough since WSI are very different from one part of the image to another, in other words, they are highly non-stationary. Labelling all the patches of one WSI could mean labelling thousands of images, which is unfeasible.

All in all, we chose this way because we believed it was going to give better results and because the doctors preferred that solution over the other for its interpretability.

\section{Data}\label{sec:data}

Now that we have described the problem as a segmentation and classification one, let's describe the datasets in which it is going to be applied and how they were obtained. The first one and the main focus of the thesis is the DigiPatics lung dataset, it is called like this because it was created under the DigiPatics project \cite{DigiPatics2022} and contains patches of lung WSI. Then, I'll describe another dataset also focused on tumour detection but of another organ, the DigiPatics breast dataset. And at the end I'll explain two more dataset: CoNSeP \cite{hovernet} and MoNuSAC \cite{8880654}. They are two public datasets related to digital pathology and nuclei segmentation and classification as well. 

\subsection{DigiPatics lung dataset}\label{sec:data_lung}

This dataset contains 85 images of 1024 by 1024 pixels referring to Hematoxilin and Eoxin stained WSI. Every patch comes from a WSI of a patient with lung tumour and it is annotated pixel-wise, that is, every pixel is either 0, 1 or 2 depending on whether it belongs to the background, to a healthy cell or to a tumoural cell. To have a better understanding of the problem at hand, look at \autoref{fig:labels} below.

\begin{figure}[ht]
  \centering
  \begin{subfigure}[b]{0.45\textwidth}
    \includegraphics[width=\textwidth]{imgs/seg.png}
    \caption{Segmentation}
    \label{fig:seg}
  \end{subfigure}
  \hfill
  \begin{subfigure}[b]{0.45\textwidth}
    \includegraphics[width=\textwidth]{imgs/overlay.png}
    \caption{Classification}
    \label{fig:class}
  \end{subfigure}
  \caption{Visualisation of the kind of labels that are needed. For each cell it is required to have its contour, as seen in the left, together with their corresponding class as seen in the right. Blue means tumoural and green non-tumoural.}
  \label{fig:labels}
\end{figure}

In order to annotate those images it is first needed to choose which images to label. One can extract hundreds of patches from surgical biopsies made by thoracotomy, and even in needle biopsies it is possible to obtain more than two hundred patches. But labelling those patches is very expensive and so they have to be chosen carefully. One important aspect to take into account is the number of patients at consideration. One thousand labelled images are of no use if they come from only one patient. There is a huge variance across WSI of different patients. For that reason we decided to annotate images from nine different patients. The selection within patient was made based on structures that seem very different from each other so that the dataset captures as much variance as possible.

Once the images were chosen, the annotation can begin. However, labelling manually each image can take days for each image. That is why we followed a semi-automated approach. To alleviate the amount of work required, an iterative procedure was designed. In a first step made by David Anglada and Feliu Formosa, the max-tree \cite{maxtree} was used to create rough segmentations of 24 patches that were later reviewed and improved by the students. Those initial labels were used to train Hovernet \cite{hovernet}, explained in detail in \autoref{sec:vision}. Using that newly trained model, 20 more images were annotated and reviewed by me. Then, Hovernet was trained again using those 44 images and used to infer the GT of 41 more patches. Only after carefully correcting all the 85 patches, would the pathologist Irene Sansano Valero start reviewing the dataset. The whole process took around 100 hours of human labour, 20 hours of GPU computation and 15 hours of expert human labour.

As a side note, this dataset is enough for the research carried out here but is definitely not enough for a production setting. Nine patients is by no means a representative sample. And there are many structures not present in the patches selected. We made our best effort to annotate a dataset that would validate our approach. Having proved its validity it remains to extend the dataset and scale the models in order to achieve a production ready model.

\subsection{DigiPatics breast dataset}\label{sec:data_breast}

Another tumour related dataset is provided by the DigiPatics group. This one was annotated by David Anglada and supervised by pathologist Teresa Soler. It contains 141 images from 4 different patients that had breast cancer. The biopsies were stained with HER2 staining. This time the dataset contains 6 different classes counting the background. I will not dive into much details of this dataset and what represents every class, the only fact I would like to mention is that domain experts on this field and the engineers collaborating with them do not think that group structure is as important here as it is in the case of lung tumours. In fact, experiments described in \autoref{subsec:gnn-xgb} and \autoref{subsec:void-gnn} show that this is the case. In \autoref{fig:mama_ex} you can see an example of the type of images that are in this dataset.

\begin{figure}[H]
    \centering
    \begin{subfigure}[b]{0.24\textwidth}
    \includegraphics[width=\textwidth]{imgs/data/p1.png}
    \caption{Patient 1}
  \end{subfigure}
  \hfill
  \begin{subfigure}[b]{0.24\textwidth}
    \includegraphics[width=\textwidth]{imgs/data/p2.png}
    \caption{Patient 2}
  \end{subfigure}
  \hfill
  \begin{subfigure}[b]{0.24\textwidth}
    \includegraphics[width=\textwidth]{imgs/data/p3.png}
    \caption{Patient 3}
  \end{subfigure}
  \hfill
  \begin{subfigure}[b]{0.24\textwidth}
    \includegraphics[width=\textwidth]{imgs/data/p4.png}
    \caption{Patient 4}
  \end{subfigure}
    \caption{Patches of the four patients that are in this database. Names are omitted for data privacy. As you can observe, images are visually very different from patient to patient even though the same staining is employed in all of them.}
    \label{fig:mama_ex}
\end{figure}

\subsection{CoNSeP dataset}\label{sec:data_consep}

The name means colorectal nuclear segmentation and phenotypes. In other words, it is about cell nuclei detection on colorectal adenocarcinoma WSIs. Yet another organ in which we are interested of automatically visualise the cells that are related to a tumour. It uses the same staining as the DigiPatics lung dataset: H\&E. The size of the images is similar: 1000x1000. And it also has many patients: 16 in total. The main difference with the other datasets so far is that it only contains 41 images. Another difference with the lung dataset is that is has 7 classes, without counting the background. Instead of classifying cells as tumour vs non-tumour the authors here decided to provide a more detailed analysis of each cell. In \autoref{fig:consep_ex} you have an example of what are the images it contains. We decided to use this dataset because it was the one used in the Hovernet article although we don't have too much insight on this type of data. The results of the experiments on this dataset seem to prove that graphs neural network are not a good choice for colorectal tissue. Nonetheless, more research is needed to confirm that hypothesis. It remains to ask experts if the group structure is important here and other approaches need to be carried out, like binarising the classes. Since in this thesis I could only talk with lung experts, I couldn't carry those experiments myself. It is left as future research.

\begin{figure}[H]
    \centering
    \includegraphics[width=\textwidth]{imgs/data/consep.png}
    \caption{Different examples of the CoNSeP dataset illustrating all the different cells that it contains. Image was taken from the original article \cite{hovernet}.}
    \label{fig:consep_ex}
\end{figure}

\subsection{MoNuSAC dataset}\label{sec:data_monusac}

As in previous subsection, the abbreviation has some meaning. It comes from Multi-Organ Nuclei Segmentation and Classification Challenge. This time it is not tumour related but it is still a nuclei segmentation and classification problem so we deemed it interesting to try our method here. The MoNuSAC database contains 294 images from 71 patients. There are several differences with respect to the other three datasets. The first one is the type of classes it has. They are not tumour related, instead cells are classified into four types: epithelial, lymphocite, macrophage and neutrophil. The second difference is the number of organs. In the previous datasets only one tumour was considered at a time. Here we have lung, breast, kidney and prostate. The third and last difference is that the WSI came from 37 hospitals instead of just one. Every WSI was selected from The Cancer Genome Data Portal\footnote{\url{https://portal.gdc.cancer.gov/} Accessed 15th May 2023} and were later labelled. Having such amount of data made this dataset quite interesting to use. It has more variety and many more images than all the datasets previously mentioned. Interestingly enough, using graph neural networks here gives better results than not using them. Experimental results will be discussed in more detailed on \autoref{sec:results}. One technical detail to remark is that images from this dataset have very different aspect ratios and sizes. To simplify the process we just upsampled or downsampled every image into having size 1024x1024 using Lanczos interpolation. An example of the images from this dataset is in \autoref{fig:monusac_ex}.

\begin{figure}[H]
    \centering
    \includegraphics[width=\textwidth]{imgs/data/monusac.png}
    \caption{Image extracted from the MoNuSAC article \cite{8880654}. It depicts various images from each of the four organs used in the creation of it.}
    \label{fig:monusac_ex}
\end{figure}

\section{Computer vision algorithms}\label{sec:vision}

In this section I will provide a brief survey about the state of the art in segmentations problems and a detailed explanation of Hovernet \cite{hovernet} which is the model used in the first phase of our method. Prior to diving into specific neural network architectures I will make a quick recap into what a convolutional neural network is. The forward pass of a single neuron from a layer of a neural network can be described as

\begin{equation}
    a_j = f(\bw^{(j)} \bx + b_j) = f\Big( \big(\sum_{i=1}^n w_i^{(j)} x_i\big) +b_j ) \Big)
\end{equation}

\noindent where $w^{j}$ is the j-th row of the weight matrix of that layer $W$, $x_i$ is the i-th entry of the input and $b_j$ is the j-th entry of the bias of that layer\footnote{\url{https://atcold.github.io/pytorch-Deep-Learning/en/week02/02-3/} Accessed 15th May 2023}. Here $f$ is any non-linearity to give the network expressivity. Now, a convolutional layer is similar in that it also has neurons that are made of a linear operator applied to the input plus a bias and a non-linearity. However, the operator is different. For a normal neural network, also called multilayer perceptron, the lineal operator can be expressed as $(W \cdot \bx)^{(j)}$ being $\cdot$ the matrix multiplication and $(\cdot)^{(j)}$ denotes the j-th entry of the vector. For a convolutional neural network, the operator is the convolution $(W * \bx)^{(j)}$. In one dimension, if $W = (w_1, w_2, w_3)^T$ is a kernel of dimension 3, then the convolution can be expressed as a matrix multiplication like so\footnote{\url{https://atcold.github.io/pytorch-Deep-Learning/en/week04/04-1/} Accessed 15th May 2023}

\begin{equation}
    W * \bx = \begin{bmatrix}
        w_1 & w_2  & w_3  & 0 & 0 & 0 & 0&\cdots  &0\\
        0 & w_1  & w_2 & w_3  & 0&0&0&\cdots &0\\
        0 & 0 & w_1 & w_2  & w_3  & 0&0&\cdots &0\\
        0 & 0 & 0& w_1  & w_2  &w_3 &0&\cdots &0\\
        0 & 0 & 0& 0 & w_1  &w_2 &w_3 &\cdots &0\\
        \vdots&&\vdots&&\vdots&&\vdots&&\vdots
    \end{bmatrix}
    \begin{bmatrix}
        x_1\\
        x_2\\
        x_3\\
        x_4\\
        \vdots\\
        x_k\\
        \vdots\\
        x_n
    \end{bmatrix}
\end{equation}

To extend it to 2D data like images, we could still use this matrix definition but it would get quite cumbersome so, instead, a simple formula can be used to describe 2D convolutions.

\begin{equation}
    (W * \bx) ^{(m,n)} = \sum_{i=-\infty}^{\infty} \sum_{j=-\infty}^{\infty} w^{(m-i, n-j)} \cdot x^{(i,j)}
\end{equation}

\noindent where the superindex notation denotes pixel coordinates and the weight matrix is centred at $(0,0)$ and has compact support, same as the image, which also has compact support meaning it is only different than zero for a finite amount of coordinates. As you may have noticed, convolutions return images as output. Therefore, we can visualise what a convolutional neural network is doing by looking at the neuron activations, which in this context are called the channels instead of neurons. In \autoref{fig:conv_ex} there is an example of two possible filters that may be learned by a convolutional neural network.

\begin{figure}[H]
    \centering
    \begin{subfigure}[b]{0.3\textwidth}
    \includegraphics[width=\textwidth]{imgs/conv/lena.png}
    \caption{Original}
  \end{subfigure}
  \hfill
  \begin{subfigure}[b]{0.3\textwidth}
    \includegraphics[width=\textwidth]{imgs/conv/lena2.jpg}
    \caption{Laplacian}
  \end{subfigure}
  \hfill
  \begin{subfigure}[b]{0.3\textwidth}
    \includegraphics[width=\textwidth]{imgs/conv/lena3.png}
    \caption{Gaussian}
  \end{subfigure}
    \caption{Example of convolution filters. At the left we have a very classical image, called Lena. On the center there is the laplacian filter applied to it using a convolution and on the right there is the gaussian filter applied to it.}
    \label{fig:conv_ex}
\end{figure}

A single convolution is not expressive enough to solve segmentation problems. Normally, several layers are required, but having too many layers has the problem of vanishing and exploding gradient which is why residual connections are needed. That was one of the ideas behind the first deep learning attempt at biomedical segmentation made by Olaf Ronneberger et al. \cite{unet}. They proposed an encoder-decoder architecture as shown in \autoref{fig:unet}.

\begin{figure}[ht]
    \centering
    \includegraphics[width=\textwidth]{imgs/unet.png}
    \caption{Original U-net architecture. It was composed by convolutional layers, pooling layers and residual connections.}
    \label{fig:unet}
\end{figure}

That architecture was improved recently with the development of transformers \cite{transformer}. In 2021 Jieneng Chen et al. invented TransUNet \cite{transunet} and later on in 2022 Jeya Maria Jose Valanarasu et al. created UNeXt \cite{unext}. I will not dive into the specifics of those architectures but rather comment on their limitations and why we couldn't use them. The key limitation is their sample efficiency. Even though transformers are more sample efficient in reinforcement learning than previous deep learning methods \cite{micheli2023transformers}, they still require a fair amount of data. In Jeya Maria Jose Valanarasu et al. \cite{unext} they used two datasets of 2594 and 647 images respectively. That is at least an order of magnitude more than what we could obtain. The other option, TransUNet \cite{transunet}, was tried in a dataset with 3779 computer tomography (CT) images, yet too much for us. Apart from that, the problem they tackled was organ segmentation, which is quite different from cell segmentation. In order to achieve better sample efficiency, a method with specific inductive biases is needed.

As explained in \autoref{sec:data}, the max-tree \cite{maxtree} can give good results at detecting cell contours while using no data at all. The algorithm used morphological properties of the cells to distinguish them from the background. That set a precedent, morphological algorithms could help us reduce the amount of data needed. Another important morphological algorithm is the watershed \cite{watershed}. It is known for creating accurate contours if the energy landscape is properly defined. Hovernet \cite{hovernet} combines both the U-net architecture with the watershed algorithm to produce cell segmentations and classify them. An overview is on \autoref{fig:hovpipe}.

\begin{figure}[ht]
    \centering
    \includegraphics[width=\textwidth]{imgs/hovpipe.png}
    \caption{Overview of the Hovernet method. It has three branches that predict different maps derived from the GT. In a post-processing step all the maps are combined using the watershed algorithm with carefully designed energy landscapes and markers.}
    \label{fig:hovpipe}
\end{figure}

Hovernet employs the same encoder-decoder architecture as U-net but it combines three different decoders with only one shared encoder. Each of the three decoders is trained to infer a different property from the GT. The NP branch separates the cells from the background, ignoring their class. The HV branch predicts horizontal and vertical distances from each pixel to the nuclei of the nearest cell. And the TP branch predicts the GT as is. Everything is trained end-to-end under one single loss function, which is shown below

\begin{align}
    \mathcal{L} &= \lambda_{mse}^{HV}\mathcal{L}_{mse}^{HV} + \lambda_{msge}^{HV}\mathcal{L}_{msge}^{HV} + \lambda_{bce}^{NP}\mathcal{L}_{bce}^{NP} + \lambda_{dice}^{NP}\mathcal{L}_{dice}^{NP} + \lambda_{bce}^{TP}\mathcal{L}_{bce}^{TP} + \lambda_{dice}^{TP}\mathcal{L}_{dice}^{TP} \\
    &= \frac{\lambda_{mse}^{HV}}{B}\left( \sum_{i=0}^{B} \| H(x_i) - h_i \|_2^2 + \sum_{i=0}^{B} \| V(x_i) - v_i \|_2^2 \right) \\
    &+ \frac{\lambda_{msge}^{HV}}{B}\left( \sum_{i=0}^{B} \| \nabla H(x_i) - gh_i \|_2^2 + \sum_{i=0}^{B} \| \nabla V(x_i) - gv_i \|_2^2 \right) \\
    &+ \frac{\lambda_{bce}^{NP}}{B} \sum_{i=0}^B \sum_{j=0}^{D} (y_{i}^{NP})_j \log (NP(x_i)_j) \\
    &+ \frac{\lambda_{dice}^{NP}}{B} \sum_{i=0}^B \left(1 - \frac{\sum_{j=0}^D NP(x_i)_j (y_{i}^{NP})_j}{\sum_{j=0}^D (NP(x_i)_j)^2 + \sum_{j=0}^D ((y_{i}^{NP})_j)^2}\right)\\
    &+ \frac{\lambda_{bce}^{TP}}{B} \sum_{i=0}^B \sum_{j=0}^{D} (y_{i}^{TP})_j \log (TP(x_i)_j) \\
    &+ \frac{\lambda_{dice}^{TP}}{B} \sum_{i=0}^B \left(1 - \frac{\sum_{j=0}^D TP(x_i)_j (y_{i}^{TP})_j}{\sum_{j=0}^D (TP(x_i)_j)^2 + \sum_{j=0}^D ((y_{i}^{TP})_j)^2}\right)
\end{align}

\noindent where all the $\lambda$ are hyperparameters, the letter $y$ denotes GT in any form, $h$ and $v$ are horizontal and vertical GT maps, $gh$ and $gv$ are the gradients of horizontal and vertical maps, $D$ is the number of pixels in any image, $B$ is the batch size, $\| \cdot \|_2^2$ is the $L_2$ norm, $H(\cdot)$ and $V(\cdot)$ are the outputs of the HV branch, $NP(\cdot)$ the output of the NP branch and $TP(\cdot)$ the output of the TP branch. This loss is particularly interesting because it combines multiple ideas. It is a mix of classification, regression and segmentation losses. Moreover, it can be considered as a second order optimisation since it is using the mean square error over the gradients. On the other side it combines the cross-entropy which is designed specially for classification problems while optimising the Dice loss which is more or less like maximising the intersection of predicted and real cells, more on that on \autoref{subsec:dice}. It is expected that optimising all the different objectives while using a single encoder is going to make that encoder extract features useful for a variety of tasks, thus generalising better.

After the model is trained, it can be used for inference in addition with a post-processing phase which consists of the watershed algorithm. This particular watershed requires an energy landscape which defines the space where the flooding is made and a marker that contains the starting points to start the flooding. Both are defined below

\begin{align}
    E &= (1 - \mathcal{S}_m(\bX)) \odot NP(\bX) \\
    M &= \text{ReLU}(NP(\bX) - \mathcal{S}_m(\bX))
\end{align}

\noindent being $E$ the energy and $M$ the marker. In those equations $\bX$ refers to the input image, ReLU is the rectified linear unit \cite{relu}, $\odot$ is the element-wise multiplication, also referred to as Hadamard product, and $\mathcal{S}_m(\bX)$ is the thresholded gradient of the HV branch as expressed here

\begin{equation}
    \mathcal{S}_m(\bX) = \max(S_x * H(\bX), S_y * V(\bX))
\end{equation}

\noindent where $S_x$ and $S_y$ are Sobel filters \cite{sobel} and $*$ is the convolution operation. The whole process can be visualised in \autoref{fig:hovpipe}. The reason for using gradient filters over the HV branch is that if the prediction is perfect, then such gradients are exactly the NP branch. Therefore, if we apply a watershed of one over the other it is expected that one branch fills the errors of the other. That is not so simple in practice, and for this post processing algorithm to work both the HV branch and the NP branch need to be expanded and contracted using a thresholding function over a threshold that was carefully selected by the authors based on empirical results.

\section{Graph neural networks}\label{sec:gnn}

Having described the state of the art for computer vision, let's introduce the state of the art of graph neural networks as well. Graph neural networks are very similar to neural networks but the key difference is that the computational graph is different for every node and it varies depending on which nodes it is connected to. GNNs as used in this thesis generate an embedding which decodes all the relevant information of that node. Most of the techniques used with neural networks can be extended to GNNs as well, like dropout \cite{dropout}, batch normalisation \cite{batchnorm} or pooling layers \cite{graph_survey}. In fact, there are more than 50 different possible architectures \cite{graph_survey} and more than 300.000 possible configurations \cite{you2021design}. However I will focus mainly on two of the most popular layers: graph convolution and graph attention. Both can be used for node classification which is what we are interested about since we are going to treat cells as nodes. More on that description in \autoref{sec:descr}.

\subsection{Graph convolution}\label{sec:gcn}

This architecture was proposed by Thomas N. Kipf et al. \cite{graphconv} in 2016 and has been cited almost ten thousands times as of this date. The main idea is to adapt the notion of convolution from images to graphs. An illustration of the concept can be seen in \autoref{fig:conv_comp}.

\begin{figure}[ht]
    \centering
    \begin{subfigure}[b]{0.4\textwidth}
        \includegraphics[width=\textwidth]{imgs/conv.png}
        \caption{2D Convolution where each pixel can be considered a node connected to all its adjacent pixels. This operation returns the weighted average of adjacent pixels for each node.}
    \end{subfigure}
    \hfill
    \begin{subfigure}[b]{0.4\textwidth}
        \includegraphics[width=\textwidth]{imgs/graph conv.png}
        \caption{Graph Convolution where the weighted average is taken with respect to adjacent nodes. There is no notion of pixels and each node has no absolute spatial coordinates.}
    \end{subfigure}
    \caption{Visualisation of 2D Convolution vs Graph Convolution taken from \cite{graph_survey}.}
    \label{fig:conv_comp}
\end{figure}

Mathematically, the graph convolution operation as expressed in \cite{graphconv} can be defined as shown below

\begin{equation}
    \bh_j^{(l+1)} = \sigma \left(\bb^{(l)} + \sum_{k \in \mathcal{N}_j} \frac{1}{c_{jk}} \bW^{(l)}\bh_k^{(l)}\right)
\end{equation}

\noindent where $\bb^{(l)}\in \R^d, \bW^{(l)} \in \R^{d\times d}$ are the bias and weights of the layer, $\mathcal{N}_j$ is the set of neighbours of node $j$, $c_{jk} = \sqrt{|\mathcal{N}_j|\cdot |\mathcal{N}_k|}$ is a normalisation factor and $\sigma$ is an activation function. The vectors $\bh_k^{(l)}$ are the hidden embeddings of the network for each layer, being $\bh_k^{(0)}$ an initial vector containing any relevant information about the node. That information can be the area of the cell, the average colour, or even a prior distribution for the class label. In the last layer, the weight matrix is of dimensions $C \times d$, where $C$ is the number of classes or $1$ if $C=2$ and the activation function is either the sigmoid for a binary problem or the softmax \cite{softmax} for a multi-class problem.

\subsection{Graph attention}

As an improvement over simply doing the average, one year after the publication of the graph convolution, Petar Veličković et al. \cite{graphatt} proposed the idea of including the attention mechanism \cite{attention} to compute a weighted average instead. This idea, which has been cited over eight thousands times, is visualised in \autoref{fig:gat}.

\begin{figure}[ht]
    \centering
    \includegraphics[width=\textwidth]{imgs/gat.png}
    \caption{On the left is the overview of the computation of the attention weights. For each adjacent node an attention weight is computed based on the similarity of their embeddings. On the right there is a visualisation about multi-head attention, which consists of concatenating the result of several attention mechanisms. The figures are taken from the original article \cite{graphatt}.}
    \label{fig:gat}
\end{figure}

More formally, the computation can be described as follows

\begin{equation}
    \bh_j^{(l+1)} = \sigma \left(\sum_{k \in \mathcal{N}_j} \alpha_{jk} \bW^{(l)}\bh_k^{(l)}\right)
\end{equation}

\noindent where $\bW^{l}\in \R^{d\times d}$ are the layer weights and $\alpha_{jk} \in \R$ are the attention weights which are defined by the following formula

\begin{equation}
    \alpha_{jk} = \frac{\exp(\text{LeakyReLU}(\ba \cdot [\bW \bh_j||\bW \bh_k]))}{\sum_{r\in \mathcal{N}_j} \exp(\text{LeakyReLU}(\ba \cdot [\bW \bh_j||\bW \bh_r]))}
\end{equation}

\noindent being $\ba \in \R^{2d'}, \bW \in \R^{d'\times d}$ two learnable projection matrices, LeakyReLU is the leaky rectified linear unit \cite{leakyrelu} and $||$ the concatenation operation. Inspired by the multi-head attention mechanism proposed in \cite{transformer}, the previous attention mechanism can be extended to $H$ heads

\begin{equation}
    \bh_j^{(l+1)} = \bigparallel_{h=1}^H \sigma \left(\sum_{k \in \mathcal{N}_j} \alpha_{jkh} \bW^{(l)}_h\bh_k^{(l)}\right)
\end{equation}

\noindent where now $\bW^{(l)}_h \in \R^{d \times Hd}$, the attention weights are different for each head and sum up to one in each head $\sum_{k\in\mathcal{N}_j}\alpha_{jkh}=1,\ \forall h$ and in the final layer heads are averaged instead of concatenated as explained in \cite{graphatt}. Notice that there is not bias as with the convolution. This is because the attention scores are thought to be similarities between nodes. The attention mechanism is just a way of averaging embeddings based on their similarity to the target node. There is no need for bias under that interpretation.

\section{XGBoost}

The last section of this chapter is devoted to give a very brief explanation of an algorithm that is going to be tangentially used in the experiments as a way to perform an ablation study to see if the graphs are really being useful or not. XGBoost \cite{xgboost} is an algorithm to perform either regression or classification over tabular data. In our case, it is going to be used over the extracted features of each cell (perimeter, area, ...) to predict its class. The XGBoost algorithm is in fact a gradient boosted tree, the distinction here comes because XGBoost is a faster implementation of such idea. To understand what a gradient boosted tree we first need to explain what a random forest is. It is a set of tree classifiers that are ensembled together using the average of their predictions. A visualisation of it is here below\footnote{\url{http://aiweb.techfak.uni-bielefeld.de/content/bworld-robot-control-software/} Accessed 25th January 2023}.

\scalebox{0.7}{
\centering
\begin{forest}
  for tree={l sep=3em, s sep=2em, anchor=center, inner sep=0.4em, fill=blue!50, circle, where level=2{no edge}{}}
  [
  Training Data, node box
  [sample and feature bagging, node box, alias=bagging, above=3em
  [,red!70,alias=a1[[,alias=a2][]][,red!70,edge label={node[above=1ex,red arrow]{}}[[][]][,red!70,edge label={node[above=1ex,red arrow]{}}[,red!70,edge label={node[below=1ex,red arrow]{}}][,alias=a3]]]]
  [,red!70,alias=b1[,red!70,edge label={node[below=1ex,red arrow]{}}[[,alias=b2][]][,red!70,edge label={node[above=1ex,red arrow]{}}]][[][[][,alias=b3]]]]
  [~~~$\dots$~,scale=2,no edge,fill=none,yshift=-3em]
  [,red!70,alias=c1[[,alias=c2][]][,red!70,edge label={node[above=1ex,red arrow]{}}[,red!70,edge label={node[above=1ex,red arrow]{}}[,alias=c3][,red!70,edge label={node[above=1ex,red arrow]{}}]][,alias=c4]]]]
  ]
  \node[tree box, fit=(a1)(a2)(a3)] (t1) {};
  \node[tree box, fit=(b1)(b2)(b3)] (t2) {};
  \node[tree box, fit=(c1)(c2)(c3)(c4)] (tn) {};
  \node[below right=0.5em, inner sep=0pt] at (t1.north west) {Tree 1};
  \node[below right=0.5em, inner sep=0pt] at (t2.north west) {Tree 2};
  \node[below right=0.5em, inner sep=0pt] at (tn.north west) {Tree $n$};
  \path (t1.south west)--(tn.south east) node[midway,below=4em, node box] (mean) {mean in regression or majority vote in classification};
  \node[below=3em of mean, node box] (pred) {prediction};
  \draw[black arrow={5mm}{4mm}] (bagging) -- (t1.north);
  \draw[black arrow] (bagging) -- (t2.north);
  \draw[black arrow={5mm}{4mm}] (bagging) -- (tn.north);
  \draw[black arrow={5mm}{5mm}] (t1.south) -- (mean);
  \draw[black arrow] (t2.south) -- (mean);
  \draw[black arrow={5mm}{5mm}] (tn.south) -- (mean);
  \draw[black arrow] (mean) -- (pred);
\end{forest}
}

Now, to pass from a random forest to a gradient boosted tree we have to substitute the idea of bagging (doing the average) to boosting. Boosting consists of incrementally add models to the ensemble that each of them predicts the error of the previous ensemble. This way every model you add operates on a smaller target variable. For classification problems, the residual is taken from the log likelihood so that it is treated as a regression problem over the real line. The process is represented on \autoref{fig:gdbt}\footnote{Image taken from \url{https://medium.com/analytics-vidhya/what-is-gradient-boosting-how-is-it-different-from-ada-boost-2d5ff5767cb2} Accessed 15th May 2023}.

\begin{figure}[H]
    \centering
    \includegraphics[width=0.8\textwidth]{imgs/conv/gdbt.png}
    \caption{Visualisation of the concept of boosting for tree models. Image taken from a medium post.}
    \label{fig:gdbt}
\end{figure}
\chapter{Problem Solving}

Having described the problem at hand as well as relevant deep learning methods it is time to join it all into one single method. In this chapter we will cover a detailed explanation of the method proposed, the metrics that will be used to compare across architectures and hyperparameters and the experiments carried out to show its usefulness.

\section{Method description}\label{sec:descr}

I have described how the cell classification problem was tackled with the help of convolutional neural networks. I have also explained two algorithms used for node classification. It is time to merge both fields. For that, we need to describe the cell classification problem as a node classification problem. What are going to be our nodes? The individual cells extracted from Hovernet. Using that computer algorithm we are going to infer the location of the cells in a patch by considering the contours in the predicted segmentation. Those contours are used just to extract what we call morphological features of the cell and its coordinates in the image. It is left to define the edges. We are going to consider two nodes (cells) to be related if they are sufficiently close. By sufficiently close it is meant that their euclidean distance is less than some previously defined amount. Apart from that, to have manageable graphs, the degree of each node is limited by only considering a small amount of nearby nodes as possible connections. This way we ensure the number of edges increases linearly with the number of nodes making our method more scalable. An example of such graph is on \autoref{fig:graph_def}. 

\begin{figure}[h]
    \centering
    \begin{subfigure}{0.3\textwidth}
      \centering
      \includegraphics[width=\textwidth]{imgs/img_ex.png}
    \end{subfigure}
    \begin{subfigure}{0.3\textwidth}
      \centering
      \includegraphics[width=\textwidth]{imgs/graph_overlay.png}
    \end{subfigure}
    \begin{subfigure}{0.3\textwidth}
      \centering
      \includegraphics[width=\textwidth]{imgs/graph_class.png}
    \end{subfigure}
    \caption{Example of an image and its associated graph. At the middle we have the nodes located at their corresponding centroids. However, a graph is an abstraction, so it may also be viewed as in the right image, since it only encodes relationships, not absolute positions. The visualization of the graph was made using Gephi \cite{gephi}.}
    \label{fig:graph_def}
\end{figure}

Given nodes and edges we can still give more information to the graph neural network. In \autoref{sec:gcn} we showed that the network can be given an initial set of features $\bh_k^{(0)}$. Those features can be anything that gives information about the cell. We decided to use the following set of descriptive features:

\begin{itemize}
    \item The area and perimeter of the cell measured in pixels. Those values should give information about the shape of the cell.
    \item The standard deviation of the values of the pixels in gray format. This magnitude is supposed to bring insights about the luminosity of the cell.
    \item The histogram of the red, green and blue colour channels. We expect it to summarise the information about the colour of the cell.
    \item A prior distribution of the class. It is computed using the output of Hovernet. Each class probability is inferred as the number of pixels of that class predicted by Hovernet divided by the total number of pixels of the cell.
\end{itemize}

\noindent Later on, in \autoref{sec:exp} an experiment is described to discover how relevant the selected features are.

\section{Hyperparameter tuning}

In our first step, Hovernet, we did not try to optimize any hyperparameter for two reasons. The first, the compute power needed is simply prohibitive. One configuration requires several hours to train. Doing cross-validation on it or trying more than 10 configurations is going to last for weeks for every dataset involved. The other reason is that the method is quite stable. Looking at the loss function during training for the train and validation dataset, we observed that after a few tens of epochs the curve converged for both datasets. Moreover, we are going to build on top of the Hovernet output, tuning Hovernet is independent from tuning the graph model on top. By showing that stacking a GNN on top of it makes the metrics improve, it is quite probable that by improving the backbone, the whole model improves as well. 

For the graph networks we will be  tuning 4 variables. The first one is the number of layers. We fixed the number of neurons on its layer to 100 and try architectures from 1 layer up to 15 layers. The second hyperparameter we tune is the dropout rate. We apply dropout after every graph layer. The third hyperparameter is batch normalisation. We consider models with and without it. And the last hyperparameter is the type of the graph layer, either convolution or attention. Having defined those, we perform a grid search over them. We train all the different configurations and evaluate them in a validation dataset. The configuration that gives the best validation score is then evaluated on the test set, and those are the metrics that are reported. Since we are optimising the type of graph layer used, I will refer to the model as GNN and not as GCN (convolution) of GAT (attention) since it could be any of those.

In the case of XGBoost the tuning is done a bit differently. We optimize over the learning rate, the maximum depth of each tree and the percentage of features visible to each tree. The number of trees is fixed at 500 for every configuration. Since XGBoost is so efficient we could perform cross validation to estimate the validation score. Concretely, we used 10 fold cross validation. Then, the configuration with the best cross validation score was tested on the test set, and those are the metrics reported.

\section{Evaluation metrics}\label{sec:metrics}

This section is going to provide a review of the most common metrics for any classification problem, either binary of multiclass. 

\subsection{Confusion Matrix}

Prior to defining any metric we have to define the concept of confusion matrix. Most of the metrics described can be expressed in terms of it. The confusion matrix is a way of measuring how precise is any method. For a binary classification problem one has positive (1) and negative (0) classes. If the model correctly predicts the positive or negative class it is called true positive and true negative. Then, if the model incorrectly predicts positive it is called false positive and if it infers negative wrongly it is denoted by false negative. Confusion matrices are typically expressed as shown below.

\begin{table}[ht]
\centering
\caption{Binary confusion matrix.}
\begin{tabular}{c c c|c|}
& & \multicolumn{2}{c}{\textbf{Predicted}} \\ \cline{3-4}
& & \multicolumn{1}{|c|}{Positive} & Negative \\ \cline{2-4}
\multirow{2}{*}{\textbf{Actual}} & \multicolumn{1}{|c|}{Positive} & $TP$ & $FP$ \\ \cline{2-4}
                     & \multicolumn{1}{|c|}{Negative} & $FN$ & $TN$ \\ \cline{2-4}
\end{tabular}
\label{table:confusion_matrix}
\end{table}

For more than two classes sometimes an adaptation is made. Instead of adding more rows and columns the matrix is built considering one class against all the others. In that case several confusion matrices are needed. Of course, one can also create a bigger confusion matrix as follows

\begin{table}[ht]
\centering
\caption{Multi-class confusion matrix. Here the terms true positive, negative and false positive, negative lack any meaning unless you consider one class against the others.}
\label{table:confusion_matrix2}
\begin{tabular}{c c c|c|c|c|c|}
& & \multicolumn{5}{c}{\textbf{Predicted}} \\ \cline{3-7}
& & \multicolumn{1}{|c|}{Class 1} & Class 2 & Class 3 & Class 4 & Class 5 \\ \cline{2-7}
\multirow{5}{*}{\textbf{Actual}} & \multicolumn{1}{|c|}{Class 1} & & & & & \\ \cline{2-7}
                     & \multicolumn{1}{|c|}{Class 2} & & & & & \\ \cline{2-7}
                     & \multicolumn{1}{|c|}{Class 3} & & & & & \\ \cline{2-7}
                     & \multicolumn{1}{|c|}{Class 4} & & & & & \\ \cline{2-7}
                     & \multicolumn{1}{|c|}{Class 5} & & & & & \\ \cline{2-7}
\end{tabular}
\end{table}

\subsection{Accuracy}\label{sec:acc}

The first metric we will be defining is the most intuitive one. It is basically the percentage of correct predictions. Using the terminology from \autoref{table:confusion_matrix} it can be expressed as 
\begin{equation}
    \frac{TP + TN}{TP + FP + TN + FN}
\end{equation}

The main disadvantage of the accuracy comes when dealing with imbalanced datasets. By predicting the class that appears the most a high accuracy can be easily achieved in those cases.

The accuracy is a binary classification metric, later on in \autoref{sec:micro} an adaptation to multi-class problems is described.

\subsection{Precision}

The accuracy requires to know how many true negatives there are. But in some problems like object detection that is not always possible. Due to how labels are constructed in some cases it is impossible to know how many true negatives can be considered, although in the case of object detection the trend seems to be changing these days \cite{kirillov2023segment}. In those cases it makes sense to define the percentage of correct predictions only within the positive class. Mathematically, precision is defined as

\begin{equation}
    P = \frac{TP}{TP + FP}
\end{equation}

This metric, however, can be easily fooled. In an image with hundreds of cells, by only predicting one true tumoural cell you achieve a precision of $100\%$. But that is a useless value since you would be missing on most relevant cells.

\subsection{Recall}

As opposed to precision, recall focuses more on what relevant values are retrieved rather than them being correct. It is defined like this

\begin{equation}
    R = \frac{TP}{TP + FN}
\end{equation}

Again, this metric can also be fooled. By predicting everything as positive you achieve $100\%$. Since this metric ignores false positives you are left with a biased metric against the negative label.

\subsection{$F_1$ Score}\label{sec:f1}

In order to take the best from precision and recall, the F-measure was proposed in 1992 at the Proceedings of the 4th conference on Message understanding \cite{10.5555/1072064}. The F-measure is defined as the harmonic mean between precision and recall.

\begin{equation}
    \frac{2 \cdot P \cdot R}{P + R} = \frac{2 \cdot TP}{2 \cdot TP + FP + FN}
\end{equation}

Nowadays it is called $F_1$ score because that measure has been extended to what is called the $F_\beta$ score.

\begin{equation}
    F_\beta = \frac{(1+\beta^2) \cdot P \cdot R}{\beta^2 \cdot P + R} = \frac{(1+\beta^2) \cdot TP}{(1+\beta^2) \cdot TP + \beta^2 \cdot FP + FN}
\end{equation}

\noindent By taking $\beta=1$ the original F-measure is obtained. This metric achieves $100\%$ when all the relevant positive samples are retrieved and only the relevant samples, therefore it is not so easily fooled. Moreover, it is less prone to suffer from class imbalance as the accuracy does.

There are three ways of extending the $F_1$ score to a multi-class classification problem. All of them involve some kind of averaging the individual $F_1$ scores computed by considering one class against all the others. 

\subsection{Macro $F_1$ Score}

The first way of averaging individual $F_1$ scores is by simply taking the arithmetic mean. If we have $n$ classes, and we call $F_1^i$ the $F_1$ score of the class $i$ against the others, then the macro $F_1$ score is

\begin{equation}
    \frac{1}{n} \sum_{i=1}^n F_1^i
\end{equation}

The main drawback of only considering this metric is that classes can be imbalanced. In fact, for multi-class problems that is the rule rather than the exception. Giving equal weights to all of the classes harms the less represented labels.

\subsection{Weighted $F_1$ Score}

As a way of solving the main drawback of the macro $F_1$ score one can deal with the weighted $F_1$ score that averages individual scores based on their support, that is, based on the number of true instances by class. Let's call $n_i$ the number of true instances of class $i$, then the weighted $F_1$ score is

\begin{equation}
    \sum_{i=1}^n \frac{n_i \cdot F_1^i}{n_1 + \dots + n_n}
\end{equation}

\subsection{Micro $F_1$ Score}\label{sec:micro}

Another way of averaging the individual $F_1$ scores is by micro-averaging. When macro-averaging, true positives, false negatives and false positives are computed per-class prior to averaging all the scores computed as defined in \autoref{sec:f1}. Micro-averaging changes the order. True positives, false negatives and false positives are first aggregated among all the classes and then the micro $F_1$ score is computed using the formula in \autoref{sec:f1}. To illustrate the computation let's consider the matrix from \autoref{table:confusion_matrix2} and let's also fill it with false positives / negatives and true positives, giving the matrix in \autoref{table:confusion_matrix3}. Now, this terminology only makes sense when splitting by class. For that reason we will denote by $TP_i$ the true positives of class $i$, and $FP_i$, $FN_i$ the false positives and negatives of same class $i$. Notice that what is a false positive for one class can be a false negative for another.

\begin{table}[ht]
\centering
\caption{Multi-class confusion matrix. The values in the diagonal are all true positives when considering one class agains the others. Depending on which class you are considering, the values considered false positives and false negatives could be interchanged.}
\begin{tabular}{c c c|c|c|c|c|}
& & \multicolumn{5}{c}{\textbf{Predicted}} \\ \cline{3-7}
& & \multicolumn{1}{|c|}{Class 1} & Class 2 & Class 3 & Class 4 & Class 5 \\ \cline{2-7}
\multirow{5}{*}{\textbf{Actual}} 
 & \multicolumn{1}{|c|}{Class 1} & $TP_1$ & $FP_1$, $FN_2$ & $FP_1$, $FN_3$ & $FP_1$, $FN_4$ & $FP_1$, $FN_5$ \\ \cline{2-7}
 & \multicolumn{1}{|c|}{Class 2} & $FP_2$, $FN_1$ & $TP_2$ & $FP_2$, $FN_3$ & $FP_2$, $FN_4$ & $FP_2$, $FN_5$ \\ \cline{2-7}
 & \multicolumn{1}{|c|}{Class 3} & $FP_3$, $FN_1$ & $FP_3$, $FN_2$ & $TP_3$ & $FP_3$, $FN_4$ & $FP_3$, $FN_5$ \\ \cline{2-7}
 & \multicolumn{1}{|c|}{Class 4} & $FP_4$, $FN_1$ & $FP_4$, $FN_2$ & $FP_4$, $FN_3$ & $TP_4$ & $FP_4$, $FN_5$ \\ \cline{2-7}
 & \multicolumn{1}{|c|}{Class 5} & $FP_5$, $FN_1$ & $FP_5$, $FN_2$ & $FP_5$, $FN_3$ & $FP_5$, $FN_4$ & $TP_5$ \\ \cline{2-7}
\end{tabular}
\label{table:confusion_matrix3}
\end{table}

The micro-average is then the sum of the diagonal divided by the sum of all the entries in the matrix. It is quite similar to how the accuracy is computed. For that reason sometimes this metric is referred to as the accuracy in multi-class problems.

\newpage
\subsection{Dice's coefficient}\label{subsec:dice}

The Dice's coefficient can be viewed as a generalisation of the $F_1$ score. Given two sets $X$ and $Y$ the Dice's coefficient is defined as 

\begin{equation}
    \frac{2\cdot |X \cap Y|}{|X| + |Y|}
\end{equation}

If we consider $X$ as the set of relevant items and $Y$ as the set of retrieved elements, we obtain the $F_1$ score. To show that, let's see what are the sets of retrieved and relevant objects. The relevant items are the sum of true positives and false negatives. The retrieved ones are the sum of true positives and false positives. The intersection is clearly just the true positives, so $|X\cap Y| = TP$ and also $|X|+|Y|=2\cdot TP + FN + FP$. Substituting into the formula for the Dice's coefficient the formula for the $F_1$ score appears.

But the Dice's coefficient can be used for more than that. It can be used as a metric for image segmentation problems. By defining $X$ as the set of pixels that belongs to a class in the ground truth and $Y$ as the set of pixels of the same class but in the predictions, the Dice's coefficient can be used for evaluating the performance of a segmentation model. 

Furthermore, it is possible to extend that measure to a loss function. All the metrics presented so far require a thresholding function at the end. That function has a discontinuity at $0.5$ but worse than that, are completely flat in the rest of the $[0,1]$ interval, which means the gradient is zero. A null gradient stops any deep learning method from using them as loss functions. The adaptation of the Dice's coefficient to a loss was made by Milletari et al. \cite{milletari2016vnet}. The idea is to take advantage from the fact that pixel class probabilities range from 0 to 1. The Dice's coefficient can be seen as a boolean operation. Therefore, the Dice's loss is a function that when given just 0s and 1s is that same boolean operation, but is also defined in the rest of the $[0,1]$ interval and not just in the extremes. To be consistent with the original notation, let's call $p_i$ to the predicted probabilities and $g_i$ to the ground truth probabilities, where $i$ ranges from $1$ to $N$ being $N$ the total number of pixels. Thus, the Dice's loss is

\begin{equation}
    D = \frac{2 \sum_{i=1}^N p_i g_i}{\sum_{i=1}^N p_i^2 + \sum_{i=1}^N g_i^2}
\end{equation}

It is clear that when $p_i \in \{0,1\} \forall i$ and $g_i \in \{0,1\} \forall i$ the result is the same as the Dice's coefficient. But in this new version, the gradient can be computed with respect to any pixel with this formula.

\begin{equation}
    \frac{\partial D}{\partial p_j} = 2\cdot \frac{g_j\left(\sum_{i=1}^N p_i^2 + \sum_{i=1}^N g_i^2\right) - 2p_j \sum_{i=1}^N p_ig_i}{\left(\sum_{i=1}^N p_i^2 + \sum_{i=1}^N g_i^2 \right)^2}
\end{equation}

\subsection{ROC AUC}

Another metric that avoids thresholding the probabilities is the Receiver Operating Characteristic Area Under the Curve. It evaluates the ordering of the probabilities instead of the actual predictions. Before diving into the details of the ROC curve, we need to first define the false positive rate.

\begin{equation}
    FPR = \frac{FP}{FP + TN}
\end{equation}

False positives and true negatives depend on the threshold selected. Using $0.5$ gives some predictions while using $-1$ give everything as positive and using $2$ returns all negatives. By changing the threshold different FPR are obtained. Moreover, the Recall, also known as true positive rate (TPR), changes when using different thresholds too. There is a balance between both of them, similar to what happened with precision and recall. Both metrics can be fooled, but not at the same time. So by changing the threshold one can measure which metric is being fooled the most. The ROC curve plots the TPR in the y axis and the FPR in the x axis. An example is on \autoref{fig:auc}.

\begin{figure}[ht]
    \centering
    \includegraphics[width=\textwidth]{imgs/auc.png}
    \caption{Example of a ROC curve.}
    \label{fig:auc}
\end{figure}

This curve always starts in the $(0,0)$ and finishes in the $(1,1)$ corresponding to thresholds $2$ and $-1$ respectively (or any other threshold that returns all negatives and all positives). Given that, the ROC AUC is, as its name states, the area under that curve. A random classifier would yield the orange line in \autoref{fig:auc} that goes from $(0,0)$ to $(1,1)$, while a perfect classifier would go from $(0,0)$ to $(0,1)$ and then to $(1,1)$. Therefore, any classifier has an AUC roughly between $0.5$ and $1$. Notice that an AUC of $0$ corresponds to an adversarial classifier, which is a perfect classifier but changes negatives with positives.

\subsection{Calibration}

To finish this section, I'll cover a different way of evaluating classifiers. Normally, the effectiveness of a classifier is measured depending on how well it classifies samples. But that has its flaws too. It may be interesting to have a way of measuring the uncertainty of a prediction. If I am going to be diagnosed with cancer I want to know how likely that is wrong. Having a $51\%$ probability of dying is not the same as having a $99.9\%$ probability. None of the previous metrics evaluates the quality of the uncertainty provided by the probabilities. Deep learning methods oftentimes suffer from overconfidence \cite{wei2022mitigating, meronen2023fixing, melotti2022reducing}. Neural networks typically provide probabilities that are close to $1$ even when the available information is not enough to be so sure about that prediction. To analyse that phenomena, reliability diagrams were invented. An example is provided in \autoref{fig:rel}. On the x axis there is the predicted probability while in the y axis is an estimation of the real probability.

\begin{figure}[ht]
    \centering
    \includegraphics[width=\textwidth]{imgs/rel.png}
    \caption{Example of a reliability diagram.}
    \label{fig:rel}
\end{figure}

More specifically, predicted probability is quantised. The x axis represents the probability distribution of the model, but we only have a finite collection of samples, so that distribution is estimated using the histogram. This means that the points in \autoref{fig:rel} in fact represents a set of samples that are predicted with very similar probabilities. The magnitude in the y axis is how many of them really are positive. As an example, suppose we have 4 samples predicted with probabilities $0.74, 0.75, 0.75, 0.76$ and only three of them are real positives. The mean predicted probability of that group is $0.75$, and the real probability is also $0.75$ since 3 out of 4 are positives. This would give a point in the black line, a perfectly calibrated point. On the other hand, consider the following example. Four samples, 3 with probability 1 and 1 with probability 0. From the first three, one is negative and two are positive. The sample with probability 0 is negative. In this case, we have a point at $(0,0)$ and another at $(1,\frac{2}{3})$. This time, we have a less calibrated prediction. Nonetheless, in both examples we have an accuracy of $75\%$. This illustrates the fact that calibration is independent from how well you classify samples. An example of an uncalibrated model is on \autoref{fig:hov_rel}.

\begin{figure}[ht]
    \centering
    \includegraphics[width=\textwidth]{imgs/hov_rel_diag.png}
    \caption{Reliability diagram of Hovernet from one of our experiments. At the right is the histogram of the predicted probabilities, clearly not a uniform distribution.}
    \label{fig:hov_rel}
\end{figure}

Reliability diagrams can be converted into a metric in the same way the ROC was made a metric by using the area under it. Here, instead of the area under the curve, the area between the model curve and a perfectly calibrated curve is taken. Furthermore, since points in that diagram can represent sets of arbitrary size the area of each bin is weighted by the number of samples in it. Where now we are calling the points bins, since in fact, each point has a width that represents the range of probabilities it takes into account. A more appropriate visualisation is on \autoref{fig:rel2}\footnote{Image taken from \url{https://github.com/EFS-OpenSource/calibration-framework/issues/17} Accessed 15th May 2023}. The metric derived is called Expected Calibration Error (ECE).

\begin{figure}[ht]
    \centering
    \includegraphics[width=\textwidth]{imgs/rel2.png}
    \caption{Another way of representing a reliability diagram. The blue colour is the real probability of a class in that group. The two red colours denote the difference with respect to the optimally calibrated classifier. The difference can be by lack or by excess, that is why there are two colours. The darker one means the real probability is higher than the predicted one. And the lighter one means the predicted probability is lower than what it should be.}
    \label{fig:rel2}
\end{figure}

\subsection{Extending metrics}

I have depicted a clear way of measuring uncertainty calibration in the previous section but it has at least one downside, it is only for binary problems. I'll devote this subsection to a way in which one can extend any binary metric to a multiclass metric. Suppose we have a metric called $M$.  Now, given a multiclass setup one can compute $M^k$, the same metric but considering the class $k$ against the others. By grouping all the other classes into one, the problem is now binary and so we can compute $M$. By averaging all the $M^k$ we get the extension of $M$. For instance, the macro $F_1$ score correspond to doing this to the $F_1$ score with equal weights to all classes. While the weighted $F_1$ score is the weighted average of the individual $F_1$ scores. However, the micro $F_1$ score, which is also called accuracy, does not correspond to the extension of the accuracy, although it is quite close. The extension of the accuracy is in fact $\frac{2}{N} \text{Micro} + \frac{N-2}{N}$ being $N$ the number of classes. The proof is straightforward. If we call $D$ the diagonal of the confusion matrix and $T$ the total sum then

\begin{align}
    Acc^k &= \frac{a_{kk} + \sum_{i,j \neq k} a_{ij}}{\sum_{i,j}a_{ij}} \\
    Acc_{extended} &= \frac{1}{N} \sum_{k} Acc^k = \frac{\sum_{k}a_{kk} + \sum_{k} \sum_{i,j\neq k} a_{ij}}{N \sum_{i,j}a_{ij}} \\
    &= \frac{2D + (N-2) T}{N T} = \frac{2}{N} \text{Micro} + \frac{N-2}{N}
\end{align}

where the intermediate step can be deduced by counting how many times does each term appears. The diagonal terms appear $N-1$ times while the rest of terms appear $N-2$ times, therefore $\sum_{k} \sum_{i,j\neq k} a_{ij} = D + (N-1)T$, concluding the proof. Surprisingly enough, the extension of the accuracy approaches 1 as the number of classes increases so it is quite a bad metric and is not used in practice, that is why the Micro $F_1$ is always preferred. In the experiments below, we will be using the extension of the ECE and call it just ECE for simplicity in the multiclass datasets.

\section{Experiments}\label{sec:exp}

Two experiments were carried out to show the usefulness of graph neural networks in the problem of cell detection and another two experiments were made to provide insights about the models involved. Results are in \autoref{sec:results}, this section is only to describe the experiments themselves. 


\subsection{GNN vs CNN}\label{subsec:gnn-cnn}

The main experiment and the one that supports the principal thesis of this work compares convolutional neural networks with graph neural networks. In order to state that GNNs are actually useful they must outperform CNNs (Hovernet) in every metric defined in \autoref{sec:metrics}. But outperforming in just one dataset is not enough. For that reason the comparison will be made using all the datasets from the original Hovernet article \cite{hovernet, gamper2020pannuke, 8880654}, and using several internal datasets of the DigiPatics project \cite{DigiPatics2022}. More concretely, the methods will be evaluated in the HER2 stained breast dataset and the H\&E stained lung dataset. For the multi-class problems, multi-class metrics will be taken into account and for binary problems, binary metrics.

\subsection{GNN vs XGBoost}\label{subsec:gnn-xgb}

Outperforming CNNs may not be due to the graph structure. It is possible that the extracted features are sufficient to improve CNN results. It may be that the edges do not contain any useful information. To account for that, GNN must be compared with node-only methods. In this case we selected XGBoost \cite{xgboost} for comparison since it has been in the leaderboard of several Kaggle competitions\footnote{\url{https://dataaspirant.com/xgboost-algorithm/}}. By comparing node-only methods with GNNs we can know if the value of GNNs reside on the extracted features or on the graph structure itself.

\subsection{Void GNNs}\label{subsec:void-gnn}

Finally, the last experiment is about knowing if the extracted features are relevant or not. We distinguish two types of features: morphological and probabilities. The first group is independent from the Hovernet output while the second is not. To see which of them are more important we train the same GNN models with different features. One with all the features, one only with morphological features and another only with probabilities. By comparing the metrics obtained we can discern which set of features is more relevant.

\subsection{Scaling CNNs}\label{subsec:scaling}

The original Hovernet article restricted their images to having 270x270 pixels. That is a very narrow view of the cells. It seems intuitive that increasing the receptive field of the model increases its performance too. But science does not understand intuition, only facts. So an experiment needs to be carried out in order to prove if increasing the receptive field is really beneficial. Apart from that, since the original weights of Hovernet are open source, we can initialise our weights with them. Therefore, there are four different models that we will call 270, 270FT, 518 and 518FT. The number indicates the resolution of the images and FT means fine-tuned. The models with FT in the name initialise their weights with the ones from the original Hovernet article. The other models initialise their weights at random. The datasets used in this experiment are only our own lung and breast datasets.
\chapter{Results}
\section{Quantitative analysis}\label{sec:results}

In what follows there will be four tables per section showing the results of each experiment for four different datasets. CoNSeP and MoNuSAC datasets are available online and more general, while DigiPatics breast and DigiPatics lung  are private datasets designed with a more specific goal in mind. Also, at every table the best results will be showcased in bold.

\subsection{GNN vs CNN}

This is the main experiment of my thesis. It showcases if using graphs improves over not using them. For two out of the three multiclass datasets it do improves. Moreover it also lowers the ECE, showing it is not only predicting better but is better calibrated. Nonetheless, the remaining dataset poses a question. Why is not working there? Two possible reasons. On the one hand, it is a small dataset with less that 30 images for training and it has 7 classes. This makes the creation of structures quite unusual, cells are more disperse in space not forming groups. And for those that do make groups, there are few sample for the algorithm to learn about it. On the other hand, it may also be that the model is overfitting. It is possible that the probabilities given to the GNN correlate with the target label more in the training that in the test set, making the model learn some patterns that are wrong. Since the other two datasets have more that 100 images, we deem proper to assume that GNNs start to outperform CNNs when provided with sufficiently enough data. Another remarkable property is that even though GNNs are worse in the consep dataset overall, they have a lower ECE, meaning they are still better calibrated.

\begin{table}[ht]
\centering
\caption{Result of the GNN vs CNN experiment.}
\begin{tabular}{c|c|c|c|c|}
  \cline{2-5}
  & Micro $F_1$ score ($\uparrow$) & Macro $F_1$ score ($\uparrow$) & Weighted $F_1$ score ($\uparrow$) & ECE ($\downarrow$) \\ \hline
\multicolumn{1}{|c|}{CNN}  & \textbf{71.11\%} & \textbf{54.06\%} & \textbf{70.39\%} & 0.0680 \\ \hline
\multicolumn{1}{|c|}{GNN}  & 64.44\% & 47.87\% & 61.42\% & \textbf{0.0539} \\ \hline
\end{tabular}
\caption{CoNSeP dataset.}

\vspace{0.5cm}


\begin{tabular}{c|c|c|c|c|}
  \cline{2-5}
  & Micro $F_1$ score ($\uparrow$) & Macro $F_1$ score ($\uparrow$) & Weighted $F_1$ score ($\uparrow$) & ECE ($\downarrow$) \\ \hline
\multicolumn{1}{|c|}{CNN}  & 82.06 \% & 68.76\% & 82.34\% & 0.0774 \\ \hline
\multicolumn{1}{|c|}{GNN}  & \textbf{88.71\%} & \textbf{69.72\%} & \textbf{89.05\%} & \textbf{0.0251}  \\ \hline
\end{tabular}
\caption{MoNuSAC dataset.}

\vspace{0.5cm}

\begin{tabular}{c|c|c|c|c|}
  \cline{2-5}
  & Micro $F_1$ score ($\uparrow$) & Macro $F_1$ score ($\uparrow$) & Weighted $F_1$ score ($\uparrow$) & ECE ($\downarrow$) \\ \hline
\multicolumn{1}{|c|}{CNN}  & 65.12\% & 40.55\% & 66.38\% & 0.3243 \\ \hline
\multicolumn{1}{|c|}{GNN}  & \textbf{70.47\%} & \textbf{42.53\%} & \textbf{71.13\%} & \textbf{0.2501} \\ \hline
\end{tabular}
\caption{DigiPatics breast dataset.}

\vspace{0.5cm}

\begin{tabular}{c|c|c|c|c|}
  \cline{2-5}
  & Accuracy & $F_1$ score & ROC AUC & ECE \\ \hline
\multicolumn{1}{|c|}{CNN}  &  &  &  &  \\ \hline
\multicolumn{1}{|c|}{GNN}  &  &  &  &  \\ \hline
\end{tabular}
\caption{DigiPatics lung dataset.}
\label{tab:gnn-cnn}
\end{table}

\newpage
\subsection{GNN vs XGBoost}

In the first experiment it was shown that GNNs outperform CNNs in some scenarios. But why? Is it due to the relations among cells or due to stacking another classifier on top of the CNN? In the former we should expect GNN to also outperform a node-only method like XGBoost. If it is the latter, then XGBoost should win. As we can see here, the answer is not crystal clear. In some cases it is better to use GNNs and in others it is not. In order to further elucidate when is the case in advance to training the models, the qualitative analysis will help give some insight. Looking at individual images will provide some keys about when graphs are a good fit.

\begin{table}[ht]
    \centering
    \caption{Result of the GNN vs XGBoost experiment.}
    \begin{tabular}{c|c|c|c|c|}
  \cline{2-5}
  & Micro $F_1$ score ($\uparrow$) & Macro $F_1$ score ($\uparrow$) & Weighted $F_1$ score ($\uparrow$) & ECE ($\downarrow$) \\ \hline
\multicolumn{1}{|c|}{XGB}  & \textbf{71.54\%} & \textbf{51.64\%} & \textbf{69.87\%} & \textbf{0.0320} \\ \hline
\multicolumn{1}{|c|}{GNN}  & 64.44\% & 47.87\% & 61.42\% & 0.0539  \\ \hline
\end{tabular}
\caption{CoNSeP dataset.}

\vspace{0.5cm}

\begin{tabular}{c|c|c|c|c|}
  \cline{2-5}
  & Micro $F_1$ score ($\uparrow$) & Macro $F_1$ score ($\uparrow$) & Weighted $F_1$ score ($\uparrow$) & ECE ($\downarrow$) \\ \hline
\multicolumn{1}{|c|}{XGB}  & 85.23\% & \textbf{76.09\%} & 85.20\% & 0.0449 \\ \hline
\multicolumn{1}{|c|}{GNN}  & \textbf{88.71\%} & 69.72\% & \textbf{89.05\%} & \textbf{0.0251} \\ \hline
\end{tabular}
\caption{MoNuSAC dataset.}

\vspace{0.5cm}

\begin{tabular}{c|c|c|c|c|}
  \cline{2-5}
  & Micro $F_1$ score ($\uparrow$) & Macro $F_1$ score ($\uparrow$) & Weighted $F_1$ score ($\uparrow$) & ECE ($\downarrow$) \\ \hline
\multicolumn{1}{|c|}{XGB}  & \textbf{78.51\%} & \textbf{47.36\%} & \textbf{79.78\%} & 0.2502   \\ \hline
\multicolumn{1}{|c|}{GNN}  & 70.47\% & 42.53\% & 71.13\% & \textbf{0.2501}   \\ \hline
\end{tabular}
\caption{DigiPatics breast dataset.}

\vspace{0.5cm}

\begin{tabular}{c|c|c|c|c|}
  \cline{2-5}
  & Accuracy & $F_1$ score & ROC AUC & ECE \\ \hline
\multicolumn{1}{|c|}{XGB}  &  &  &  &  \\ \hline
\multicolumn{1}{|c|}{GNN}  &  &  &  &  \\ \hline
\end{tabular}
\caption{DigiPatics lung dataset.}
    \label{tab:gnn-xgb}
\end{table}

\newpage
\subsection{Scaling CNNs}

Deleting last layer weights and retraining can give better performance, as shown below in \autoref{tab:consep-scaling}. 270FT performs better than 270 and 518FT better than 518 for the CoNSeP dataset. This aligns perfectly with the results from \cite{zhou2022fortuitous}. They observe that reinitialising weights and retraining can boost performance. In our setup this translates to fine-tuning a model in the same dataset it was pretrained. Also, you may have noticed that the metrics for 518FT are slightly different than those in the GNN vs CNN experiment even though they are the same model. That is because we trained the same model twice converging to slightly different checkpoints. I provide both metrics to showcase that the ordering is the same using either of both metrics, thus proving training is sufficiently stable.

Same result is obtained for MoNuSAC dataset. Fine-tuning the checkpoint trained on CoNSeP helps obtain better results when applied as initialisation for the models trained on MoNuSAC dataset which comes from a totally different distribution. The features learned by the encoder seems to generalise well to this other dataset. However, the results in the breast dataset differ from the ones in the CoNSeP and MoNuSAC datasets. In that case using a pretrained checkpoint didn't perform well at neither resolution. This may be because the local minima found for the CoNSeP dataset is far away from the nearest minima in the DigiPatics breast dataset, making a random initialisation a better method. 

Another remarkable fact is that using a field of view of 518 pixels instead of 270 gives better metrics no matter the initialisation nor the dataset. We would have liked to further try a bigger field of view. But training the 518 models required more than 20 GB of GPU VRAM. Scaling to 1030x1030 images would require near 80 GB of GPU VRAM, which is only feasible using A100 or H100, which cost more than 10000€. Using CPU offloading was also not an option since that technique trades memory for time, typically increasing by 100 the time required. The models required around 4 hours to train. This means one experiment may take up to 2 weeks with a bigger field of view, which is clearly prohibitive. With our resources, 518 was the maximum we could achieve.

\begin{table}[ht]
    \centering
    \caption{Result of the Scaling CNNs experiment.}
    \begin{tabular}{c|c|c|c|}
  \cline{2-4}
  & Micro $F_1$ score ($\uparrow$) & Macro $F_1$ score ($\uparrow$) & Weighted $F_1$ score ($\uparrow$) \\ \hline
\multicolumn{1}{|c|}{270}  & 54.20\% & 36.21\% & 54.62\% \\ \hline
\multicolumn{1}{|c|}{270FT}  & 66.73\% & 49.01\% & 65.96\% \\ \hline
\multicolumn{1}{|c|}{518}  & 56.75\% & 37.76\% & 59.45\% \\ \hline
\multicolumn{1}{|c|}{518FT}  & \textbf{71.02\%} & \textbf{53.83\%} & \textbf{70.52\%} \\ \hline
\end{tabular}
\caption{CoNSeP dataset.}
\label{tab:consep-scaling}


\vspace{0.5cm}

\begin{tabular}{c|c|c|c|}
  \cline{2-4}
  & Micro $F_1$ score ($\uparrow$) & Macro $F_1$ score ($\uparrow$) & Weighted $F_1$ score ($\uparrow$)  \\ \hline
\multicolumn{1}{|c|}{270}  & 76.46\% & 56.20\% & 77.04\% \\ \hline
\multicolumn{1}{|c|}{270FT}  & 77.97\% & 62.84\% & 77.93\% \\ \hline
\multicolumn{1}{|c|}{518}  & 78.57\% & 58.64\% & 79.61\% \\ \hline
\multicolumn{1}{|c|}{518FT}  & \textbf{82.02\%} & \textbf{70.40\%} & \textbf{82.09\%} \\ \hline
\end{tabular}
\caption{MoNuSAC dataset.}

\vspace{0.5cm}

\begin{tabular}{c|c|c|c|c|}
  \cline{2-4}
  & Micro $F_1$ score ($\uparrow$) & Macro $F_1$ score ($\uparrow$) & Weighted $F_1$ score ($\uparrow$) \\ \hline
\multicolumn{1}{|c|}{270}  & 70.04\% & 38.22\% & 68.43\%  \\ \hline
\multicolumn{1}{|c|}{270FT}  & 48.55\% & 21.57\% & 34.44\% \\ \hline
\multicolumn{1}{|c|}{518}  & \textbf{72.36\%} & \textbf{45.64\%} & \textbf{75.80\%} \\ \hline
\multicolumn{1}{|c|}{518FT}  & 68.78\% & 43.27\% & 71.58\% \\ \hline
\end{tabular}
\caption{DigiPatics breast dataset.}

\vspace{0.5cm}

\begin{tabular}{c|c|c|c|c|}
  \cline{2-5}
  & Accuracy & $F_1$ score & ROC AUC & ECE \\ \hline
\multicolumn{1}{|c|}{270}  &  &  &  &  \\ \hline
\multicolumn{1}{|c|}{270FT}  &  &  &  &  \\ \hline
\multicolumn{1}{|c|}{518}  &  &  &  &  \\ \hline
\multicolumn{1}{|c|}{518FT}  &  &  &  &  \\ \hline
\end{tabular}
\caption{DigiPatics lung dataset.}
    \label{tab:scaling}
\end{table}

\newpage
\subsection{Void GNNs}

Another way of seeing if the good performance of graphs is due to the information in the edges or in the attributes of the nodes, is to train the graphs models with and without the attributes. If there is no performance difference, then edges are relevant. Otherwise, it is less relevant. Moreover, there is a set of attributes that depends on the layers behind, the probabilities. To discern if GNNs are working cause they are stacked above or because there is a graph structure we also train the models with and without the probabilities. We observe that in the dataset that XGBoost did not outperform GNNs the GNN trained with no features do in fact perform better than the model with features. And also, using probabilities not only did not improve but worsened the result. However, in the two datasets that XGBoost did give better results we can see that using probabilities gives a benefit over using other types of features. This tells us that MoNuSAC has more structure than CoNSeP and DigiPatics breast datasets. In DigiPatics breast GNNs gave better results than CNN but it was due to stacking. For MoNuSAC it was because the graph is indeed a good way of modelling the problem. 

\begin{table}[ht]
    \centering
    \caption{Result of the Void GNNs experiment.}
    \begin{tabular}{c|c|c|c|c|}
  \cline{2-5}
  & Micro $F_1$ score ($\uparrow$) & Macro $F_1$ score ($\uparrow$) & Weighted $F_1$ score ($\uparrow$) & ECE ($\downarrow$) \\ \hline
\multicolumn{1}{|c|}{Full}  & 64.44\% & \textbf{47.87\%} & 61.42\% & 0.0539  \\ \hline
\multicolumn{1}{|c|}{Probabilities}  & \textbf{64.93\%} & 47.37\% & \textbf{61.63\%} & \textbf{0.0494} \\ \hline
\multicolumn{1}{|c|}{Morphological}  & 52.11\% & 29.42\% & 48.19\% & 0.0582 \\ \hline
\multicolumn{1}{|c|}{Void}  & 47.97\% & 28.71\% & 47.44\% & 0.0653 \\ \hline
\end{tabular}
\caption{CoNSeP dataset.}

\vspace{0.5cm}

\begin{tabular}{c|c|c|c|c|}
  \cline{2-5}
  & Micro $F_1$ score ($\uparrow$) & Macro $F_1$ score ($\uparrow$) & Weighted $F_1$ score ($\uparrow$) & ECE ($\downarrow$) \\ \hline
\multicolumn{1}{|c|}{Full}  & 88.71\% & 69.72\% & 89.05\% & 0.0251  \\ \hline
\multicolumn{1}{|c|}{Probabilities}  & 82.59\% & \textbf{73.43\%} & 82.60\% & 0.0784 \\ \hline
\multicolumn{1}{|c|}{Morphological}  & \textbf{92.01\%} & 69.83\% & \textbf{92.02\%} & \textbf{0.0405} \\ \hline
\multicolumn{1}{|c|}{Void}  & 90.60\% & 71.00\% & 90.61\% & 0.0470 \\ \hline
\end{tabular}
\caption{MoNuSAC dataset.}

\vspace{0.5cm}

\begin{tabular}{c|c|c|c|c|}
  \cline{2-5}
  & Micro $F_1$ score ($\uparrow$) & Macro $F_1$ score ($\uparrow$) & Weighted $F_1$ score ($\uparrow$) & ECE ($\downarrow$) \\ \hline
\multicolumn{1}{|c|}{Full}  & \textbf{70.47\%} & \textbf{42.53\%} & \textbf{71.13\%} & 0.2501  \\ \hline
\multicolumn{1}{|c|}{Probabilities}  & 67.94\% & 40.83\% & 68.91\% & 0.2666 \\ \hline
\multicolumn{1}{|c|}{Morphological}  & 63.05\% & 30.48\% & 61.26\% & 0.2676 \\ \hline
\multicolumn{1}{|c|}{Void}  & 68.31\% & 36.03\% & 67.90\% & \textbf{0.2358} \\ \hline
\end{tabular}
\caption{DigiPatics breast dataset.}

\vspace{0.5cm}

\begin{tabular}{c|c|c|c|c|}
  \cline{2-5}
  & Accuracy & $F_1$ score & ROC AUC & ECE \\ \hline
\multicolumn{1}{|c|}{Full}  &  &  &  &  \\ \hline
\multicolumn{1}{|c|}{Probabilities}  &  &  &  &  \\ \hline
\multicolumn{1}{|c|}{Morphological}  &  &  &  &  \\ \hline
\multicolumn{1}{|c|}{Void}  &  &  &  &  \\ \hline
\end{tabular}
\caption{DigiPatics lung dataset.}
    \label{tab:void-gnn}
\end{table}

\newpage
\subsection{CNNs metrics in detail}

In all the experiments above I provided metrics only for 1-1 matchings of cells. That is a fair way of comparing CNN to GNN since GNN can only improve predictions on 1-1 matchings. However, it does not show the full picture. The CNN can miss cells or predict cells that do not exist. For that reason I here provide the same metrics as above but adding one extra class: the background. It never has true positives, only false positives and false negatives. Thus, it will almost always be worse than the metrics above given. For a counter example on when it is not worse, look at the next paragraph. The bigger the gap, the more cells it is missing or incorrectly predicting. Nevertheless, such gap does not alter any of the conclusions. In this project we did not try to narrow this gap because we consider the classification problem to be inherently more difficult than the segmentation problem. We believe improving the segmentation problem is simply a matter of more data and bigger models. If you think about it, you can detect cells without formal knowledge, it is just a matter of geometry and color. But recognising where are the tumours located? That requires more than 10 years of training to humans, and even in that case experts still have doubts. Therefore we directed our efforts toward improving the classification, not the segmentation.

You may have noticed that in the DigiPatics breast dataset the macro $F_1$ score did not worsen when adding the background. It seems counterintuitive since we are adding a class with no true positives, only false positives and false negatives. But, the key to why this happens is purely technical. This is the global confusion matrix for the test set in the DigiPatics breast dataset:

\[
\begin{bmatrix}
0 & 95 & 151 & 116 & 9 & 425 \\
26 & 100 & 14 & 1 & 0 & 64 \\
518 & 752 & 1136 & 819 & 17 & 445 \\
25 & 6.0 & 54 & 336 & 28 & 5 \\
0 & 0 & 0 & 0 & 0 & 0 \\
509 & 48 & 18 & 3 & 1 & 2672
\end{bmatrix}
\]

\noindent Why is this matrix problematic? Well, there is one class which does not appear at all in the ground truth. The reason why this causes the problem is because the two macro $F_1$ scores are computed differently. The "With Background" metrics are computed using the confusion matrix and a custom function I designed. The "Without Background" metrics are computed using pairs of labels and the sklearn f1\_score function. Both functions are coded properly, that is not the problem. But in my implementation of the metric, I coded a function that estimates the number of classes based on the classes with support in the ground truth, and also ignoring the zero class. However, the sklearn function estimates the number of classes as the class with the maximum label. This is making one metric being divided by 4 and another being divided by 5. If we adjust the sklearn given metric to consider 4 classes instead of 5 we get a macro $F_1$ score of $50.69\%$ which is bigger than $44.56\%$ as expected. The reason I don't adjust this metric in the other tables is because the GNN methods are also evaluated using the sklearn function so the comparison is fair as it is. I just left the metrics below unchanged to showcase how a simple decision in design of an algorithm or method can affect the final conclusions, even creating mathematically impossible situations.

\begin{table}[ht]
\centering
\caption{Hovernet evaluated with and without background in four different datasets.}
\begin{tabular}{c|c|c|c|}
  \cline{2-4}
  & Micro $F_1$ score & Macro $F_1$ score & Weighted $F_1$ score \\ \hline
\multicolumn{1}{|c|}{With Background}  & 48.38\% & 49.58\% & 57.84\% \\ \hline
\multicolumn{1}{|c|}{Without Background}  & \textbf{71.11\%} & \textbf{54.06\%} & \textbf{70.39\%} \\ \hline
\end{tabular}
\caption{CoNSeP dataset.}

\vspace{0.5cm}


\begin{tabular}{c|c|c|c|}
  \cline{2-4}
  & Micro $F_1$ score & Macro $F_1$ score & Weighted $F_1$ score \\ \hline
\multicolumn{1}{|c|}{With Background}  & 53.29\% & 43.95\% & 67.67\% \\ \hline
\multicolumn{1}{|c|}{Without Background}  & \textbf{82.06\%} & \textbf{68.76\%} & \textbf{82.34\%} \\ \hline
\end{tabular}
\caption{MoNuSAC dataset.}

\vspace{0.5cm}

\begin{tabular}{c|c|c|c|}
  \cline{2-4}
  & Micro $F_1$ score & Macro $F_1$ score & Weighted $F_1$ score \\ \hline
\multicolumn{1}{|c|}{With Background}  & 50.57\% & \textbf{44.56\%} & 57.89\% \\ \hline
\multicolumn{1}{|c|}{Without Background}  & \textbf{65.12\%} & 40.55\% & \textbf{66.38\%}  \\ \hline
\end{tabular}
\caption{DigiPatics breast dataset.}

\vspace{0.5cm}

\begin{tabular}{c|c|c|c|c|c|c|c|}
  \cline{2-7}
  & Accuracy & $F_1$ score & ROC AUC & Micro $F_1$ & Macro $F_1$ & Weighted $F_1$ \\ \hline
\multicolumn{1}{|c|}{With Background}  & NA & NA & NA &  &  & \\ \hline
\multicolumn{1}{|c|}{Without Background}  &  &  &  & NA & NA & NA \\ \hline
\end{tabular}
\caption{DigiPatics lung dataset.}
\label{tab:gnn-cnn}
\end{table}

\section{Qualitative analysis}
\chapter{Conclusions}

Digital pathology can be very difficult to tackle. Our experiments have showed that there is not a single method that works properly in every setup. Each organ and stain requires their own specific and carefully designed model. For lung and H\&E stain the method proposed not only works better than previous state of the art models but our experiments have proven that it is properly leveraging extra information hidden to computer vision algorithms and tabular methods like XGBoost. Domain expertise was key for designing the algorithm. By talking to real experts on the topic I was able to discover how important the neighbourhood of a cell was to classify it as tumoural or not. Since in digital pathology data is so scarce, discovering such inductive biases can be very valuable. A simple vision transformer would easily solve this problem with billions of images. But we have hundreds. That is why the method presented in this thesis is so valuable, because of its sample efficiency.
\chapter{Future work}

The obvious lines of research here are improving the CNN backbone, improving the GNN head or training everything end-to-end instead of having two-phases as in this work. All of them would bring better metrics and more robust classification methods. However, the obvious is not always the best. Progress is made by impressive discoveries that few people expected. That is why I want to propose a different way of advancing science in this field. The classification paradigm has served us well, but a new paradigm is emerging nowadays. Multimodal models are on the rise. Image captioning models like BLIP \cite{li2022blip} or CLIP \cite{radford2021learning} have gained popularity in the last years. They provide a way of opening the black boxes that are neural networks. Captioning an image is just one step behind a model explaining itself and the reasoning about the result. Why not make a model explain cancer? Expert pathologist train new physicians showing them images and explaining them with voice or text. Why not train models the same way? Obviously those models won't be better than the experts, but they can outscale them. One person can only view a small amount of patients in a life. A model can be trained with billions of cases in months and can infer the result of millions of new cases in days given enough computational resources. So that is the non-obvious future work: Image captioning for medical image analysis.

I'll use the rest of the page to describe the roadmap to achieving the goal of medical images explaining themselves. First step: data. That is the most important part of the whole project. No data no models. Typically foundational models require billions of image-text pairs, or if only trained with text they require trillions of tokens. Such amount of medical data is not available yet, although it would be possible if experts were into it. But I'll assume physicians don't really want to spend time creating datasets (which is more or less my experience dealing with doctors). The good news is that one can fine-tune a foundational model with smaller amounts of data. With thousands of image-text pairs would be enough for a production ready model. To obtain such pairs pathologist would be required to think loud. That's it. We can automatically transcribe audio into text, so it is enough for them to record their voice while working. The next step is the model. In the previous paragraph I have shown two methods of jointly training a latent text-image space. But we can achieve much more than that, we want to speak to the images. Something like what is done in Mini-GPT4 \cite{zhu2023minigpt4}. Right now there is a wide variety of language models to be used as backbones. Stability AI recently released StableLM \footnote{\url{https://github.com/Stability-AI/StableLM}}, and we have the LLaMa family of models too \footnote{\url{https://ai.facebook.com/blog/large-language-model-llama-meta-ai/}}. We are not running out of open source LLMs anytime soon. Using them is a huge boost in performance and sample efficiency. Moreover, there is also DINOv2 \footnote{\url{https://dinov2.metademolab.com/}} which can be used as image encoder to further reduce the burden of training. We just need to merge everything and finetune it with the appropiate data. Sounds easy, right?
\printbibliography[heading=bibintoc]
\appendix
\chapter{Sustainability and costs}

This project was carried out by a single student, which means the cost of it was very low. However, a project that depends on students is not a sustainable one since students are highly underpaid. For that reason I would like to provide a quantitative analysis on the real costs of carrying out this work. 

Let's start counting the hours of work required to just build the dataset. Even after using a semi-automatic procedure to label the data, many hours were needed to end up with 85 labelled images. Roughly 100 hours of human labour and 15 hours of compute power were employed to create segmentations. Another 13 hours of expert human labour were required to review the class annotations by the physician. Now, let's translate that into money. The first 100 hours required very low training, just the ability to paint circles. A fair salary for that can be 10€/h (if outsourced it could be less than 2€/h). That translates into a cost of 1000€. The compute power requires GPUs that consume at most 350W. Let's estimate the CPU, memory and GPU consumption by 300W on average, which means $4.5$kWh were consumed. That means $0.70$€ if the price of electricity is $0.15$€/kWh. And the expert work is typically paid more, let's estimate 20€/h which means 260€ in total. The total cost of creating the database would be $1260.70$€. If we divide by the number of images we have a total of $14.83$€ per image. 

Second, the salary of the technical workers. A project like this normally requires a software developer, a data engineer and a project manager. Depending on the experience the salaries may range. For simplicity, let's consider the software developer is junior (30K/year), the data engineer is senior (50K/year) and the project manager has nearly five years of experience (80K/year). Assuming the project is carried out in just one year, and taking into account the extra 30\% that needs to be payed to the social security, the total cost in salaries would be $208000$€.

Third, putting a machine learning model in production is more than just creating it. It needs to be maintained, and someone has to care about the drift. Models performs worse when time passes because the data distribution is not fixed. New patients means new training is needed. Hospitals would need to train their technicians to perform retrainings and physicians to relabel. The cost of relabelling is already estimated: $14.83$€ per image. Remains to estimate the cost of training technicians. 385\$ or 350€ per person is what a course on machine learning would cost\footnote{\url{https://www.ml.school/c/start-here} Accessed 8th April 2023}.

Finally, to show that the project is sustainable, it has to generate some value. I believe in open science and medicine. Putting a price in other people's lives is not ethical, so privatising this project is not an option. But someone has to pay the cost, in this case, the government. A project like this is only sustainable if public opinion is in favour of it. More data is needed to exactly estimate the social value this will bring. In my opinion, this product will reduce physician workloads, which is beneficial. By reducing the workload, the diagnostic process can be accelerated, and receiving faster diagnostics is something the public opinion will definitely be interested about.

\chapter{The problem of merging cells}

One of the problems that appeared when using Hovernet was broken cells. In many cases, big cells were predicted as various small cells. It is visually appalling and that is why we designed an algorithm to merge broken cells. The result is on \autoref{fig:morph_}.

\begin{figure}[ht]
\begin{subfigure}[b]{1\textwidth}
    \centering
    \includegraphics[width=\textwidth]{imgs/morph1.png}
    \caption{Before}
    \label{fig:morph1}
\end{subfigure}
\begin{subfigure}[b]{1\textwidth}
    \centering
    \includegraphics[width=\textwidth]{imgs/morph2.png}
    \caption{After}
    \label{fig:morph2}
\end{subfigure}
\caption{Example of cells that were not fully detected by Hovernet. After applying our algorithm all the smaller cells were merged into one.}
\label{fig:morph_}
\end{figure}

The algorithm leverages techniques of mathematical morphology. If we call $X_1$ the image predicted and $X_2$ the image that identifies the background with the value zero and the rest with one, then, the image $(\delta (X_1) - X_1)  \cdot X_2$ contains the frontiers of the cells, being $\delta$ a dilation. An illustration of the process is on \autoref{fig:morph}.

\begin{figure}[ht]
    \centering
    \includegraphics[width=0.5\textwidth]{imgs/morphInflam.png}
    \caption{Top-left is the original image, top-right is the dilation, bottom-left the gradient and bottom-right the masking which is the final result.}
    \label{fig:morph}
\end{figure}

The problem here is that different frontiers may end up with the same identifier. $X_1$ contains the identifier of each cell in every pixel of such cell. But, $(\delta (X_1) - X_1)  \cdot X_2$ contains the differences of identifiers in the position of the frontier. Differences of different pairs may end up with the same value. E.g.: $4-3 = 2-1$. To solve that, it is needed to properly set the identifiers so that each pair can be uniquely identified by the difference between the maximum and the minimum. One simple solution to that is to use powers of two. More concretely, applying the function $n \mapsto 2^n$ to the identifiers. This way $2^n-2^m$ is a function that can be inverted back to the pair $(n,m)$ where $n>m$. However, we can have up to 1500 cells, requiring identifiers up to $2^{1500}$ is clearly unfeasible and is not compute optimal since to identify pairs we only need $1500 \choose 2$ $=1.124.250$ identifiers. The best function I have found so far is $n \mapsto n^5$, using other simple lower degree polynomials doesn't uniquely identify the difference for values up to 1500 \cite{4567383}.

\chapter{TumourKit}

The whole thesis dealt with the theory and the results of the experiments. Behind all that there was a lot of software needed to make it all work. I decided to build a python library that other people could use to replicate my experiments and design new ones. The library is called TumourKit. The code is available on GitHub\footnote{\url{https://github.com/Jerry-Master/lung-tumour-study}} under the Affero GPLv3 license. It is tested on Ubuntu and Windows for python versions 3.8, 3.9 and 3.10. It has more than ten thousands lines of code, counting comments. For a more detailed explanation on how to use it and what is offered, please, read the docs\footnote{\url{https://lung-tumour-study.readthedocs.io/en/latest/}}.

\chapter{Soft Labels}

One of the problems we encountered when reviewing the lung dataset was that the expert was not sure which label to give to some cells. It was solved by simply ignoring those cells since in practice it didn't really matter because we are interested in percentages and a cell with no label can be removed from the numerator and denominator. However, we do preserve that information in the dataset because it could be used to train the models with soft labels. Even though we didn't try it, we expect it could help further calibrate the resulting probabilities. For that reason, in this appendix I will be explaining how to adapt all the methods presented in the thesis to soft labels. 

The easiest ones to adapt are the graph neural networks. They are trained using the softmax, so they are already compatible with soft labels. It is only needed to substitute the one hot encoded vector of labels to a vector of probabilities and everything works exactly the same.

The XGBoost method requires a little trick. Although XGBoost also uses softmax for training, current implementations threshold the label so providing probabilities is of no use since the soft labels will be made into hard labels. Nonetheless, XGBoost supports a weight for each row in the dataset. This makes it possible to use the probabilities as weights. The trick consists of repeating rows one time for each class and assign that class probability to that repeated row weight. The idea was taken for an online forum\footnote{\url{https://stackoverflow.com/a/66481600}} and the explanation behind how this works is that XGBoost multiplies the gradient and hessian by the weight and not by the label, as explained in another online forum\footnote{\url{https://stats.stackexchange.com/a/365555/378093}}. 

The trickiest method to adapt is HoVerNet. Having soft labels does not affect neither the HV branch nor the NP branch but it does affect the NC branch. That branch is trained using the softmax and the dice loss. The first one is quite easy to adapt as described for the graph neural networks. However, the dice loss requires more thinking. In Zifu Wang et al. \cite{wang2023dice} they propose a way of extending the dice loss for soft labels with promising results. On a more technical and practical level, modifying the codebase to support this metric is too much of a hurdle given the expected benefits. But it is interesting on a theoretical perspective to know of the existence of such possibility. In the future, when it becomes a more refined technique with wider support on the main libraries, it may be a good way of training better models.

\end{document}
