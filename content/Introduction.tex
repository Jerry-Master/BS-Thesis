\chapter{Introduction}
\section{When technology meets histology}

This project was born due to the necessity of alleviating physicians and researchers workload. On a typical day, our experts have to go through a wide variety of whole slide images (WSI). Depending on the task at hand they sometimes need to estimate the proportion of tumoural cells with respect to healthy ones. How wonderful would it be if a machine could perform that task for them. That is the purpose of this thesis, to automatically estimate the percentage of tumoural cells with respect to non-tumoural ones and facilitate the physician job.

In the past, WSI were typically watched through the lenses of a microscope \cite{Chen2011}, what is termed as optical microscopy. With the development of new technologies, the field has become more and more digitalised \cite{Kumar2020}, now being called digital pathology. This means that between when a biopsy procedure is made and when the specialist watches it, there is a period of time required to digitalise it. Taking advantage of that already needed interval, other computational methods can be applied prior to the experts receiving the images, enhancing the physician user experience while working without the need for them to wait for the algorithm to give the result.

\section{Why graphs?}

In a previous thesis, the same problem was tackled using a vision-only approach \cite{upcommons353765}. Initially, I was asked to improve on that method. After several months working on the problem I began to notice the principal flaw (which is also the principal feature) of convolutional neural networks: an inductive bias towards locality \cite{DBLP:journals/corr/CohenS16a}. That means deep neural networks classify cells based on their immediate morphological properties. However, having met with pathologist Irene Sansano Valero made it clear that cells were considered being tumoural or not depending on their surroundings. Visually identical cells may be classified differently if their neighbourhood is different. It was made clear that a different approach was needed. Another kind of inductive bias was required, a relational inductive bias. One that classified cells based on their relationships with nearby cells and not only on their individual properties. This is the exact kind of inductive bias graph neural networks provide \cite{DBLP:journals/corr/abs-2104-13478}. 