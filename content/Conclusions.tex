\chapter{Conclusions}

Digital pathology can be very difficult to tackle. Our experiments have showed that there is not a single method that works properly in every setup. Each organ and stain requires their own specific and carefully designed model. For lung and H\&E stain the method proposed not only works better than previous state of the art models but our experiments have proven that it is properly leveraging extra information hidden to computer vision algorithms and tabular methods like XGBoost. Domain expertise was key for designing the algorithm. By talking to real experts on the topic I was able to discover how important the neighbourhood of a cell was to classify it as tumoural or not. Since in digital pathology data is so scarce, discovering such inductive biases can be very valuable. A simple vision transformer would easily solve this problem with billions of images. But we have hundreds. That is why the method presented in this thesis is so valuable, because of its sample efficiency.